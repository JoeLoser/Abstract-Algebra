\documentclass{article}
\usepackage{amsmath, amsthm, amssymb}
\usepackage{parskip} % so I don't have to use \noindent everywhere
\usepackage{mathtools}
\usepackage[none]{hyphenat} % don't split up words that flow onto next line
\usepackage{microtype,todonotes}
\usepackage[a4paper,text={16.5cm,25.2cm},centering]{geometry}
\usepackage[compact,small]{titlesec}
\usepackage{polynom} % Polynomial long division
\setlength{\parskip}{1.2ex}
\setlength{\parindent}{0em}

% For addition and multiplication tables
\usepackage{dcolumn}
\newcolumntype{2}{D{.}{}{2.0}}

% To prevent orphans and widows (https://en.wikipedia.org/wiki/Widows_and_orphans)
\clubpenalty = 10000
\widowpenalty = 10000
\usepackage{kpfonts}
\usepackage[T1]{fontenc}

\title{Abstract Algebra Homework 10}
\author{\textit{Joe Loser}}
\date{April 24, 2016}
\begin{document}
\maketitle

This problem set includes problems $3b, 5b, 12$, and an extra problem from section $17.4$.

3) Use the division algorithm to find $q(x)$ and $r(x)$ such that $a(x) = q(x)b(x) + r(x)$ with $\deg r(x) < \deg b(x)$.

3b) $a(x) = 6x^4-2x^3+x^2-3x+1, b(x) = x^2 + x - 2$ in $\mathbb{Z}_7[x]$

\underline{Solution}: Performing long division, we have  $$\polylongdiv{6x^4-2x^3+x^2-3x+1}{x^2+x-2}$$

So we have that
\begin{align*}
	a(x) &= (6x^2-8x+21)(x^2+x-2) - 40x + 43 \\
	&= (6x^2+6x)(x^2+x-2) + 2x + 1.
\end{align*}

Thus $q(x) = 6x^2 + 6x$ and $r(x) = 2x + 1$. \qed \\

5) Find all of the zeros of the given polynomial.

5b) $p(x) := 3x^3-4x^2-x+4$ in $\mathbb{Z}_5$

\underline{Solution}: We first find a root of $p(x)$ by looking at $p(0), p(1), p(2), p(3)$, and $p(4)$. We see that
$\\p(0) = 4 \\
p(1) = 3 - 5 + 4 = 2 \\
p(2) = 24 - 16 + 2 = 10 \equiv 0 \\
p(3) = 81-36+1 \equiv 1-1+1 \equiv 1$ and \\
$p(4) = 3(64)-64 = 2(64) = 128 \equiv 3$. \\

So $x = 2$ is a root of $p(x)$. Performing the long division, we have that
$$ \polylongdiv{3x^3-4x^2-x+4}{x-2} $$
Note that $10 \equiv 0$ so $r(x) = 0$. Thus 
\begin{equation}
	p(x) = (3x^2+2x+3)(x-2)
\end{equation}

To further see that we cannot factorize this any more, we look to see the roots of $q(x) := (3x^2+2x+3)$ in $\mathbb{Z}_5$. We have that 
$\\q(0) = 3 \\
q(1) = 8 \equiv 3 \\
q(2) = 12 + 4 + 3 \equiv 19 \equiv 4 \\
q(3) = 3(9) + 6 + 3 = 27 + 9 = 36 \equiv 1 \\
q(4) = 3(16) + 8 + 3 \equiv 3(1) + 3 + 3 \equiv 4$ \\

So $q(x)$ cannot be factorized any much. Thus we see that the only factorization of $p(x)$ in $\mathbb{Z}_5$ is as seen in equation $(1)$. Hence the only root of $p(x)$ is $x = 2$. \qed \\

12) If $F$ is a field, show that $F[x_1, \cdots, x_n]$ is an integral domain.

\underline{Proof}: 

\end{document}