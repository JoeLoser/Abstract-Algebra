\documentclass{article}
\usepackage{times, amsfonts}
\usepackage{amsmath, amsthm, amssymb}
\usepackage{parskip} % so I don't have to use \noindent everywhere
\usepackage{mathtools}
\usepackage[none]{hyphenat} % don't split up words that flow onto next line

\title{Abstract Algebra Homework 5}
\author{\textit{Joe Loser}}
\date{February 28, 2016}
\begin{document}
\maketitle

This problem set includes problems from sections $5.3$ and $6.4$ as well as an extra problem from lecture: in particular, problem $31$ and $34$ from $5.3$, $5h$ and $8$ from section $6.4$. \\

31) For $\alpha$ and $\beta$ in $S_n$, define $a \sim b$ if there exists an $\sigma \in S_n$ such that $\sigma \alpha \sigma^{-1} = \beta$. Show that $\sim$ is an equivalence relation on $S_n$.

\underline{Proof}: To show $\sim$ is an equivalence relation, we need to show it is reflexive, symmetric, and transitive. 

(i) Let $\alpha \in S_n$. Then there exist $\sigma \in S_n$ such that $\sigma \alpha \sigma^{-1} = \alpha$, i.e. $\sigma = 1_S$. This can be seen by first left multiplying by $\sigma^{-1}$. So we have the following:
\begin{align*}
	\sigma^{-1} \sigma \alpha \sigma^{-1} &= \sigma^{-1} \alpha \\
	\alpha \sigma^{-1} &= \sigma^{-1} \alpha \\
	\alpha &= \sigma^{-1} \alpha \sigma \\
	\alpha &= \sigma^{-1} \alpha (\sigma^{-1})^{-1}
\end{align*}
Hence, $\sigma = \sigma^{-1}$ and the identity works for such $\sigma$. Therefore, $a \sim a$ and we have that $\sim$ is reflexive.

(ii) Let $\alpha, \beta \in S_n$. If $\alpha \sim \beta$ then there exist $\sigma \in S_n$ such that $\sigma \alpha \sigma^{-1} = \beta$. We want to show this implies that $b \sim a$ -- that is, $\sigma \beta \sigma^{-1} = \alpha$. By left multiplying by $\sigma^{-1}$, we have the following:
\begin{align*}
	\sigma^{-1}  \sigma \alpha \sigma^{-1} &= \sigma^{-1} \beta \\
	\alpha \sigma^{-1} &= \sigma^{-1} \beta \\
	\alpha \sigma^{-1} \sigma &= \sigma^{-1} \beta \sigma \\
	\alpha &= \sigma^{-1} \beta \sigma
\end{align*}
Now, we rewrite $\sigma$ as $(\sigma^{-1})^{-1}$ which yields
$$\alpha = \sigma^{-1} \beta (\sigma^{-1})^{-1}.$$ 
Therefore $\beta \sim \alpha$ and hence $\sim$ is symmetric.

(iii) Suppose $\alpha \sim \beta$ and $\beta \sim \gamma$. Then there exist $\sigma_1$ and $\sigma_2 \in S_n$ such that
\begin{align}
	\sigma_1 \alpha \sigma_1^{-1} &= \beta \\
	\sigma_2 \beta \sigma_2^{-1} &= \gamma
\end{align}
We want to show $\alpha \sim \gamma$, i.e. $\sigma_1 \alpha \sigma_1^{-1} = \gamma$. Multiplying equations (1) and (2) yields:
\begin{align*}
	\sigma_2 \sigma_1 \beta \alpha \sigma_1^{-1} \sigma_2^{-1} &= \gamma \beta
\end{align*}
and by canceling the $\beta$ term yields:
\begin{align*}
	\sigma_2 \sigma_1 \alpha \sigma_1^{-1} \sigma_2^{-1} &= \gamma
\end{align*}
which can be written as 
$$(\sigma_2 \sigma_1) \alpha (\sigma_2 \sigma_1)^{-1} = \gamma.$$

Thus we have shown that $\sim$ is transitive.

Therefore, $\sim$ is reflexive, symmetric, and transitive. Hence, it is an equivalence relation on $S_n$. \qed \\

34) If $\alpha$ is even, prove that $\alpha^{-1}$ is also even. Does a corresponding result hold if $\alpha$ is odd?

\underline{Proof}: Let $\alpha$ be an even permutation. Then $\alpha$ can be written as a product of an even number of transpositions: 
$$\alpha = \tau_1 \tau_2 \ldots \tau_m.$$
So $\alpha$ is even if and only if $m$ is even. Then
\begin{align*}
	\alpha^{-1} &= (\tau_1 \tau_2 \ldots \tau_m)^{-1} \\
	&= \tau_m^{-1} \tau_{m-1}^{-1} \ldots \tau_2^{-1} \tau_1^{-1}
\end{align*}
Note that for any transposition (2-cycle) $\tau$, we have that $\tau^{-1} = \tau$. Thus, 
\begin{equation}
	\alpha^{-1} = \tau_m \ldots \tau_2 \tau_1.
\end{equation}

In equation (3), if $m$ is even, then $\alpha^{-1}$ is even. A similar argument (by symmetry) can be shown for $m$ odd. If $m$ is odd, then this shows that $\alpha^{-1}$ is also odd. \qed \\

\underline{Alternative Proof}: Note that $(1)$ is even and hence $\alpha \alpha^{-1}$ is also even where $\alpha \alpha^{-1}$ is the product of the transpositions of $\alpha$ and $\alpha^{-1}$. Then this shows that
$$ \text{even number = number transpositions for } \alpha + \text{number transpositions for }\alpha^{-1}$$
since $(1)$ is an even number. Thus, $\alpha$ is even/odd if and only if $\alpha^{-1}$ is even/odd. \qed \\

5h) List the left and right cosets for $H = \{ (1), (123), (132) \}$ in $S_4$.

\underline{Solution}: Let $G = S_4$ and $H$ as above. Then, by Lagrange's Theorem -- for a finite group $G$ with $H < G$, the number of \textit{distinct} left cosets (and right cosets -- same number) is equal to
\begin{align*}
	[G:H] &= \frac{\lvert G \rvert}{\lvert H \rvert} \\
	&= \frac{4!}{3} \\
	&= \frac{24}{3} \\
	&= 8.
\end{align*}

We begin by noting that $S_4$ is a set whose order is $24$ and contains the following elements: \\
$\{ (1), (12), (13), (14), (23), (24), (34), \\
(142), (143),  (134), (132), (124), (123), (243), (234) \\
(12)(34), (14)(23), (13)(24), \\
(1423), (1432), (1324), (1342), (1243), (1234) \}.$

Let $G = S_4$. To find the left cosets of H under G, we want to keep picking $g \in G$, multiply it by $H$ and examine the sets that arise. Recall that $gH = \{gh \ \vert \ h \in H \}$ is the left coset of $H$ in $G$ with respect to $g$. Similarly, $Hg = \{ hg \ \vert \ h \in H \}$ is the right coset of $H$ in $G$ with respect to $g$. 

For the left cosets, we have the following:
\begin{enumerate}
   \item $(1)H = \{(1)(1), (1)(123), (1)(132) \} = \{(1), (123), (132) \}$
   \item $(14)H = \{(14)(1), (14)(123), (14)(132)\} = \{(14), (1234), (1324) \}$
   \item $(23)H = \{(23)(1), (23)(123), (23)(132)\} = \{(23), (13), (12) \}$
   \item $(24)H = \{(24)(1), (24)(123), (24)(132)\} = \{(24), (1423), (1342) \}$
   \item $(34)H = \{(34)(1), (34)(123), (34)(132)\} = \{(34), (1243), (1432) \}$
   \item $(124)H = \{(123)(1), (124)(123), (124)(132)\} = \{(124), (14)(23), (134) \}$
   \item $(142)H = \{(142)(1), (142)(123), (142)(132)\} = \{(142), (234), (13)(24) \}$
   \item $(143)H = \{(143)(1), (143)(123), (143)(132)\} = \{(143), (12)(34), (243) \}$
\end{enumerate}
Note that there are $8$ left cosets, each of order $3$. Each of them is disjoint as well and make up the original group $G$ which has $24$ elements. Further computations can show for example, that $(132)H = (1)H, (12)H = (23)H$, and more. But, we know we are done as we have found $8$ distinct left cosets which is all Lagrange's Theorem guarantees us. Now let's work on the right cosets. \pagebreak

For the right cosets, we have the following:
\begin{enumerate}
   \item $H(1) = \{(1)(1), (123)(1), (132)(1) \} = \{(1), (123), (132) \}$
   \item $H(14) = \{(1)(14), (123)(14), (132)(14)\} = \{(14), (1423), (1432) \}$
   \item $H(23) = \{(1)(23), (123)(23), (132)(23) \} = \{ (23), (12), (13) \}$
   \item $H(24) = \{(1)(24), (123)(24), (132)(24) \} = \{ (24), (1243), (1324) \}$
   \item $H(34) = \{(1)(34), (123)(34), (132)(34) \} = \{ (34), (1234), (1342) \}$
   \item $H(124) = \{(1)(124), (123)(124), (132)(124) \} = \{ (124), (13)(24), (243) \}$
   \item $H(142) = \{(1)(142), (123)(142), (132)(142) \} = \{ (142), (143), (14)(23) \}$
   \item $H(234) = \{(1)(234), (123)(234), (132)(234) \} = \{ (234), (12)(34), (134) \}$
\end{enumerate}

Note, again, that there are $8$ right cosets and that these are not the same as the left cosets. They do still partition the group $G$ into $8$ cosets, each of $3$ elements. \qed \\

8) Use Fermat's Little Theorem to show that if $p = 4n + 3$ is prime, there is no solution to the equation $x^2 \equiv -1 \ \pmod p$.

\underline{Proof}: Recall that Fermat's Little Theorem states that if $p$ is prime and $p \nmid a$ then $a^{p-1} \equiv 1 \ \pmod p$.

Suppose that $x^2 \equiv -1 \pmod p$. Then $x \neq 0 \pmod p$. So, by Fermat's Little Theorem, we have that $x^{p-1} \equiv 1 \pmod p$. Therefore, 
\begin{align*}
	x^{p-1} &\equiv x^{4n+3-1} \\
	&\equiv x^{4n+2} \\
	&\equiv x^{2 \cdot (2n+1)} \\
	&\equiv (x^{2})^{2n+1} \\
	&\equiv -1 \pmod p
\end{align*}

Since $x^{p-1} \equiv 1 \pmod p$ and $x^{p-1} \equiv -1 \pmod p$, we have that $1 \equiv -1 \pmod p$. So $2 \equiv 0 \pmod p$. This implies $p = 2$. However, since $p = 4n + 3$, we have that
$$2 = 4n + 3 \implies -1 = 4n \implies n = -\frac{1}{4}.$$ We have reached a contradiction now though since while there are infinitely many primes of the form $4n+3$, typically the notion is that $n \in \mathbb{N}$ and hence cannot be less than $0$. Thus, $p \neq 2$ and there are no solutions to the equation $x^2 \equiv -1 \pmod p$. \qed \\

E1) Show $63$ is not prime using Fermat's Little Theorem. 

\underline{Solution}: To begin, we want to find the first power, say $x$, such that $2^x > 63$. Clearly $x = 6$ since $2^6 = 64 > 63$. In fact, note that $2^6 \equiv 1 \pmod {63}$. Now we want to write $62$ using the Division Algorithm: namely, $62 = 10 \cdot 6 + 2$. Then in $\pmod {63}$,
\begin{align*}
	2^{62}  &= (2^6)^{10} \cdot 2^2 \\
	&= (1)^{10} \cdot 4 \\
	&= 4
\end{align*} 
by using the fact that $2^{6} \equiv 1 \pmod {63}$.
So $2^{62} \equiv 4 \pmod {63} \not\equiv 1 \pmod {63}$. Thus, $63$ is not prime by Theorem 6.19 (Fermat's Little Theorem) which states that for any $p$ prime, integer $a$ such that $p \nmid a$
$$a^{p-1} \equiv 1 \pmod p. $$
As we have shown, for $a = 2$ arbitrarily, we reached that $2^{62} \equiv 4 \pmod {63}$. Hence $63$ is not prime. \qed \\ 

\end{document}