\documentclass{article}
\usepackage{times, amsfonts}
\usepackage{amsmath, amsthm, amssymb}

\title{Abstract Algebra Homework 1}
\author{\textit{Joe Loser}}
\date{January 22, 2016}

\begin{document}
\maketitle

\noindent This problem set includes problems from sections 1.3 and 2.3. \\
1.3: Problems 24c and 29 \\
2.3: Problems 15b, 18, and 26. \\

\noindent 24c)
Let $f: X \to Y$ be a map with $A_1, A_2 \subset X$ and $B_1, B_2 \subset Y.$ Prove
\begin{center}
$ f^{-1} (B_1 \cup B_2)  = f^{-1}(B_1) \cup f^{-1}(B_2)$
\end{center}
where $f^{-1}(B) = \{\,x \in X : f(x) \in B \,\}.$ \\

\noindent
\underline{Proof}: First, we will show $f^{-1} (B_1 \cup B_2) \subset f^{-1}(B_1) \cup f^{-1}(B_2).$ If $x \in f^{-1}(B_1 \cup B_2)$, then $f(x) \in B_1 \cup B_2.$ So, either $f(x) \in B_1$ or $f(x) \in B_2.$ Therefore, $x \in f^{-1}(B_1)$ or $x \in f^{-1}(B_2).$  Hence, $x \in f^{-1}(B_1) \cup f^{-1}(B_2)$, i.e. $f^{-1} (B_1 \cup B_2) \subset f^{-1}(B_1) \cup f^{-1}(B_2).$ \\

\noindent
Next, we will show that $f^{-1}(B_1) \cup f^{-1}(B_2) \subset f^{-1} (B_1 \cup B_2).$ Let $x \in f^{-1}(B_1) \cup f^{-1}(B_2).$ Then $x \in f^{-1}(B_1)$ or $x \in f^{-1}(B_2)$. So, either $f(x) \in f(B_1)$ or $f(x) \in f(B_2).$ Hence, $f(x) \in B_1 \cup B_2.$ Therefore, $x \in f^{-1}(B_1 \cup B_2).$ \\

\noindent
Putting it all together, we have shown that $f^{-1} (B_1 \cup B_2) = f^{-1}(B_1) \cup f^{-1}(B_2).$ \\

\noindent
29) Define a relation on $\mathbb{R}^{2} \setminus \{\,(0,0)\,\} $ by letting $(x_1, y_1) \sim (x_2, y_2)$ if there exists a nonzero real number $\lambda$ such that $(x_1, y_1) = (\lambda x_2, \lambda y_2).$ Prove that $\sim$ defines an equivalence relation on $\mathbb{R}^{2} \setminus \{\,(0,0)\,\}.$ What are the corresponding equivalence classes? This equivalence relation defines the projective line, which is very important in geometry. \\

\noindent
\underline{Proof}: In order to show $\sim$ is an equivalence relation, we need to show it is reflexive, symmetric, and transitive. \\\\
i) \underline{Reflexive}: Since $x = 1 \cdot x$ and $y = 1 \cdot y$, $(x, y) \sim (x, y)$ for $(x, y) \in \mathbb{R}^{2}.$ \\
ii) \underline{Symmetric}: Suppose $(x_1, y_1) \sim (x_2, y_2).$ Then $x_1 = \lambda x_2$ and $y_1 = \lambda y_2$ for some $\lambda \in \mathbb{R} \setminus \{\,(0,0)\,\}.$ Then, $\lambda^{-1} \in \mathbb{R} \setminus \{\,(0,0)\,\}$ and $x_2 = \lambda^{-1} x_1$ and $y_2 = \lambda^{-1} y_1.$ Therefore, $(x_2, y_2) \sim (x_1, y_1).$ \\
iii) \underline{Transitive}: Suppose $(x_1, y_1) \sim (x_2, y_2)$ and $(x_2, y_2) \sim (x_3, y_3).$ We want to show that $(x_1, y_1) \sim (x_3, y_3).$ So, we have that $x_1 = \lambda x_2, y_1 = \lambda y_2, x_2 = \alpha x_3$ and $y_2 = \alpha y_3$ for some $\lambda, \alpha \in \mathbb{R} \setminus \{\,(0,0)\,\}.$ Therefore, $\lambda \alpha \in \mathbb{R} \setminus \{\,(0,0)\,\}$ and $x_1 = \lambda \alpha x_3$ and $y_1 = \lambda \alpha y_3$. Hence, $(x_1, y_1) \sim (x_3, y_3).$ \\

\noindent
Since $\sim$ is reflexive, symmetric, and transitive, it is an equivalence relation.  The corresponding equivalence class of $(x, y)$ is the line passing through $(x, y)$ and the origin. \\

\noindent
15b) For the following pair of numbers $a$ and $b$, calculate $gcd(a, b)$ and find integers $r$ and $s$ such that $gcd(a, b) = ra + sb. $ \\

\noindent
\underline{Solution}: Using the Euclidean Algorithm, we have that:
\begin{align*}
234 &= 165(1) + 69 \\
165 &= 69(2) + 27 \\
69 &= 27(2) + 15 \\
27 &= 15(1) + 12 \\
15 &= 12(1) + 3 \\
12 &= 3(4) + 0 \\
\end{align*}

\noindent
Reversing our steps, $3$ divides $12$, $3$ divides $15$, $3$ divides $27$, $3$ divides $69$, $3$ divides $165$, and $3$ divides $234.$  If $d$ were another common divisor of $234$ and $165$, then $d$ would also have to divide $3$. Therefore, $gcd(234, 165) = 3.$\\\\
Working backwards through the above sequence of equations, we can  find integers $r$ and $s$ that satisfy
$$gcd(a,b) = gcd(234, 165) = 3 = 234r + 165s.$$ So, we have the following:

\begin{align*}
3 &= 15 + (-1) (12) \\
&= [69 - 2(27)] + (-1)[27 + (-1)(15)] \\
&= [69 - 2(27)] + (-1)(27) + 15 \\
&= 69 + (-3)(27) + 69 - 27(2) \\
&= 2(69) + (-5)(27) \\
&= 2[234 - 165] + (-5)[165 + (-2)(69)] \\
&= 234(2) + 165(-2) + 165(-5) + 10(69) \\
&= 234(2) + 165(-7) + 10(69) \\
&= 234(2) + 165(-7) + 10[234 + (-1)(165)] \\
&= 234(2) + 165(-7) + 234(10) + 165(-10) \\
&= 234(12) + 165(-17) \\
\end{align*}

\noindent
Hence, $r = 12, s = -17$ satisfies $234r + 165s = 3.$ \\

\noindent
18) Let $a, b \in \mathbb{Z}$ such that $gcd(a, b) = 1$. Let $r, s$ be integers such that $ar + bs = 1$. Prove that 
\begin{center}
$gcd(a, s) = gcd(r, b) = gcd(r, s) = 1.$
\end{center}

\noindent
\underline{Proof}: Let $d := gcd(a, s).$ We want to show that $d = 1.$ By definition of gcd, we have that $d \vert a$ and $d \vert s$. Therefore, $d \vert (ar + bs).$ However, $(ar + bs) = 1$, so $d \vert 1.$ Hence, the only option is for $d = 1.$ \\

\noindent
A similar argument can be made for the remaining pairs of numbers: $(r, b)$ and $(r, s)$. It will conclude with $d \vert 1$ and hence the only option is for $d$ to be equal to $1.$ \\

\noindent
26) Prove that $gcd(a, c) = gcd(b, c) = 1$ if and only if $gcd(ab, c) = 1$ for integers $a, b$ and $c.$ \\

\noindent
\underline{Proof}: \\
$\rightarrow:$ First, we will show that $gcd(a, c) = gcd(b, c) = 1$ implies $gcd(ab, c) = 1.$ Note that $1 \vert ab$ and $1 \vert c.$ Then, it suffices to show there exists integers $x$ and $y$ such that

\begin{center}
$abx + cy = 1$ 
\end{center}
Since $gcd(a, c) = gcd(b, c) = 1$, by definition of gcd, there exists integers $k, l, m$ and $n$ such that

\begin{center}
$ak + cl = 1$ and 
$bm + cn = 1$ 
\end{center}

\noindent
Multiplying these two equations by one another yields the following:
\begin{center}
$abkm + ackn + cblm + ccln = 1$
\end{center}
which can be factored as such:
\begin{center}
$ab(km) + c(akn + blm + cln) = 1$
\end{center}

\noindent
Hence, $x = km, y = akn + blm + cln.$ \\

\noindent
$\leftarrow$: Next, to prove that $gcd(ab, c) = 1$ implies $gcd(a, c) = gcd(b, c) = 1$, notice that by assumption, there exists integers $x$ and $y$ such that
\begin{center}
$abx + cy = 1$
\end{center}

\noindent
This can be written as $a(bx) + cy = 1$ and $b(ax) + cy = 1$.  Hence, there exists integers $k', l', m', n'$ such that

\begin{center}
$ak' + cl' = 1$ and $bm' + cn' = 1$
\end{center}

\noindent
This implies that $gcd(a, c) = gcd(b, c) = 1$. \\

\noindent
Therefore, we have shown that $gcd(a, c) = gcd(b, c) \iff gcd(ab, c) = 1.$



\end{document}