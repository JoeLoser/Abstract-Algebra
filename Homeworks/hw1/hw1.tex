\documentclass{article}
\usepackage{amsmath, amsthm, amssymb}
\usepackage{parskip} % so I don't have to use \noindent everywhere
\usepackage{mathtools}
\usepackage[none]{hyphenat} % don't split up words that flow onto next line
\usepackage{microtype,todonotes}
\usepackage[a4paper,text={16.5cm,25.2cm},centering]{geometry}
\usepackage[compact,small]{titlesec}
\setlength{\parskip}{1.2ex}
\setlength{\parindent}{0em}
\DeclareMathOperator{\lcm}{lcm}

% To prevent orphans and widows (https://en.wikipedia.org/wiki/Widows_and_orphans)
\clubpenalty = 10000
\widowpenalty = 10000
\usepackage{kpfonts}
\usepackage[T1]{fontenc}

\title{Abstract Algebra Homework 1}
\author{\textit{Joe Loser}}
\date{January 22, 2016}

\begin{document}
\maketitle

\noindent This problem set includes problems from sections 1.3 and 2.3. \\
1.3: Problems 24c and 29 \\
2.3: Problems 15b, 18, and 26. \\

24c) Let $f: X \to Y$ be a map with $A_1, A_2 \subset X$ and $B_1, B_2 \subset Y.$ Prove
$$ f^{-1} (B_1 \cup B_2)  = f^{-1}(B_1) \cup f^{-1}(B_2)$$
where $f^{-1}(B) = \{ x \in X : f(x) \in B \}.$ \\

\underline{Proof}: First, we will show $f^{-1} (B_1 \cup B_2) \subset f^{-1}(B_1) \cup f^{-1}(B_2).$ If $x \in f^{-1}(B_1 \cup B_2)$, then $f(x) \in B_1 \cup B_2.$ So, either $f(x) \in B_1$ or $f(x) \in B_2.$ Therefore, $x \in f^{-1}(B_1)$ or $x \in f^{-1}(B_2).$  Hence, $x \in f^{-1}(B_1) \cup f^{-1}(B_2)$, i.e. 
$$f^{-1} (B_1 \cup B_2) \subset f^{-1}(B_1) \cup f^{-1}(B_2).$$

Next, we will show that $f^{-1}(B_1) \cup f^{-1}(B_2) \subset f^{-1} (B_1 \cup B_2).$ 
Let $x \in f^{-1}(B_1) \cup f^{-1}(B_2).$ Then $x \in f^{-1}(B_1)$ or $x \in f^{-1}(B_2)$. So, either $f(x) \in f(B_1)$ or $f(x) \in f(B_2).$ Hence, $f(x) \in B_1 \cup B_2.$ Therefore
$$x \in f^{-1}(B_1 \cup B_2).$$ 

Putting it all together, we have shown that $f^{-1} (B_1 \cup B_2) = f^{-1}(B_1) \cup f^{-1}(B_2).$ \qed \\

29) Define a relation on $\mathbb{R}^{2} \setminus \{\,(0,0)\,\} $ by letting $(x_1, y_1) \sim (x_2, y_2)$ if there exists a nonzero real number $\lambda$ such that $(x_1, y_1) = (\lambda x_2, \lambda y_2).$ Prove that $\sim$ defines an equivalence relation on $\mathbb{R}^{2} \setminus \{\,(0,0)\,\}.$ What are the corresponding equivalence classes? This equivalence relation defines the projective line, which is very important in geometry. \\

\underline{Proof}: In order to show $\sim$ is an equivalence relation, we need to show it is reflexive, symmetric, and transitive.

i) \underline{Reflexive}: Since $x = 1 \cdot x$ and $y = 1 \cdot y$, $(x, y) \sim (x, y)$ for $(x, y) \in \mathbb{R}^{2}.$ \\

ii) \underline{Symmetric}: Suppose $(x_1, y_1) \sim (x_2, y_2).$ Then $x_1 = \lambda x_2$ and $y_1 = \lambda y_2$ for some $\lambda \in \mathbb{R} \setminus \{0\}.$ Then, $\lambda^{-1} \in \mathbb{R} \setminus \{0\}$ and $x_2 = \lambda^{-1} x_1$ and $y_2 = \lambda^{-1} y_1.$ Therefore, $(x_2, y_2) \sim (x_1, y_1).$ \\

iii) \underline{Transitive}: Suppose $(x_1, y_1) \sim (x_2, y_2)$ and $(x_2, y_2) \sim (x_3, y_3).$ We want to show that $(x_1, y_1) \sim (x_3, y_3).$ So, we have that $x_1 = \lambda x_2, y_1 = \lambda y_2, x_2 = \alpha x_3$ and $y_2 = \alpha y_3$ for some $\lambda, \alpha \in \mathbb{R} \setminus \{ 0 \}.$ Therefore, $\lambda \alpha \in \mathbb{R} \setminus \{ 0 \}$ and $x_1 = \lambda \alpha x_3$ and $y_1 = \lambda \alpha y_3$. Hence, $(x_1, y_1) \sim (x_3, y_3).$ \\

Since $\sim$ is reflexive, symmetric, and transitive, it is an equivalence relation.  The corresponding equivalence class of $(x, y)$ is the line passing through $(x, y)$ and the origin, but omitting the origin. \qed \pagebreak

15b) For the following pair of numbers $a$ and $b$, calculate $gcd(a, b)$ and find integers $r$ and $s$ such that $gcd(a, b) = ra + sb. $ \\

\underline{Solution}: Using the Euclidean Algorithm, we have that:
\begin{align*}
	234 &= 165(1) + 69 \\
	165 &= 69(2) + 27 \\
	69 &= 27(2) + 15 \\
	27 &= 15(1) + 12 \\
	15 &= 12(1) + 3 \\
	12 &= 3(4) + 0
\end{align*}

Reversing our steps, $3$ divides $12$, $3$ divides $15$, $3$ divides $27$, $3$ divides $69$, $3$ divides $165$, and $3$ divides $234.$  If $d$ were another common divisor of $234$ and $165$, then $d$ would also have to divide $3$. Therefore, $gcd(234, 165) = 3.$\\\\
Working backwards through the above sequence of equations, we can  find integers $r$ and $s$ that satisfy
$$\gcd(a,b) = \gcd(234, 165) = 3 = 234r + 165s.$$ So, we have the following:
\begin{align*}
	3 &= 15 + (-1) (12) \\
	&= [69 - 2(27)] + (-1)[27 + (-1)(15)] \\
	&= [69 - 2(27)] + (-1)(27) + 15 \\
	&= 69 + (-3)(27) + 69 - 27(2) \\
	&= 2(69) + (-5)(27) \\
	&= 2[234 - 165] + (-5)[165 + (-2)(69)] \\
	&= 234(2) + 165(-2) + 165(-5) + 10(69) \\
	&= 234(2) + 165(-7) + 10(69) \\
	&= 234(2) + 165(-7) + 10[234 + (-1)(165)] \\
	&= 234(2) + 165(-7) + 234(10) + 165(-10) \\
	&= 234(12) + 165(-17)
\end{align*}

Hence, $r = 12, s = -17$ satisfies $234r + 165s = 3.$ \qed \\

18) Let $a, b \in \mathbb{Z}$ such that $\gcd(a, b) = 1$. Let $r, s$ be integers such that $ar + bs = 1$. Prove that 
$$\gcd(a, s) = \gcd(r, b) = \gcd(r, s) = 1.$$

\underline{Proof}: Let $d := \gcd(a, s).$ We want to show that $d = 1.$ By definition of $\gcd$, we have that $d \vert a$ and $d \vert s$. Therefore, $d \vert (ar + bs).$ However, $(ar + bs) = 1$, so $d \vert \pm 1$. Since $d > 0$ the only option is for $d = 1.$ \\

A similar argument can be made for the remaining pairs of numbers: $(r, b)$ and $(r, s)$. It will conclude with $d \vert \pm1$ and again, since $d > 0$ we must have that $d = 1$. \qed \\

26) Prove that $\gcd(a, c) = \gcd(b, c) = 1$ if and only if $\gcd(ab, c) = 1$ for integers $a, b$ and $c.$ \\

\underline{Proof}: \\
$\rightarrow:$ First, we will show that $\gcd(a, c) = \gcd(b, c) = 1$ implies $\gcd(ab, c) = 1.$ Note that $1 \vert ab$ and $1 \vert c.$ Then it suffices to show there exists integers $x$ and $y$ such that
$$abc + cy = 1$$
by Corollary A6.

Since $\gcd(a, c) = \gcd(b, c) = 1$, by definition of $\gcd$, there exists integers $k, l, m$ and $n$ such that
\begin{center}
$ak + cl = 1$ and 
$bm + cn = 1$ 
\end{center}

\noindent
Multiplying these two equations by one another yields the following:
$$abkm + ackn + cblm + ccln = 1$$
which can be factored as such:
$$ab(km) + c(akn + blm + cln) = 1$$

Hence, $x = km, y = akn + blm + cln.$ \\

$\leftarrow$: Next, to prove that $\gcd(ab, c) = 1$ implies $\gcd(a, c) = \gcd(b, c) = 1$, notice that by assumption, there exists integers $x$ and $y$ such that
$$abx + cy = 1.$$

This can be written as $a(bx) + cy = 1$ and $b(ax) + cy = 1$.  Hence, there exists integers $k', l', m', n'$ such that
\begin{center}
	$ak' + cl' = 1$ and $bm' + cn' = 1. $
\end{center}

By Corollary A6, this implies that $\gcd(a, c) = \gcd(b, c) = 1$. \\

Therefore, we have shown that $\gcd(a, c) = \gcd(b, c) \iff \gcd(ab, c) = 1.$ \qed \\

\end{document}