\documentclass{article}
\usepackage{amsmath, amsthm, amssymb}
\usepackage{parskip} % so I don't have to use \noindent everywhere
\usepackage{mathtools}
\usepackage[none]{hyphenat} % don't split up words that flow onto next line
\usepackage{microtype,todonotes}
\usepackage[a4paper,text={16.5cm,25.2cm},centering]{geometry}
\usepackage[compact,small]{titlesec}
\setlength{\parskip}{1.2ex}
\setlength{\parindent}{0em}
\DeclareMathOperator{\lcm}{lcm}

% To prevent orphans and widows (https://en.wikipedia.org/wiki/Widows_and_orphans)
\clubpenalty = 10000
\widowpenalty = 10000
\usepackage{kpfonts}
\usepackage[T1]{fontenc}

\title{Abstract Algebra Homework 6}
\author{\textit{Joe Loser}}
\date{March 7, 2016}
\begin{document}
\maketitle

This problem set includes problems from sections $11.3$, $9.3$, and $10.3$.

8) If $G$ is an abelian group and $n \in \mathbb{N}$, show that $\phi : G \to G$ defined by $g \mapsto g^{n}$ is a group homomorphism. 

\underline{Proof}: Let $a, b \in G$. It is easy to see that $\phi : G \to G$ is well-defined for all $g \in G$. To show $\phi$ is a group homomorphism we need to show that $\phi(ab) = \phi(a)\phi(b)$ for all $a, b \in G$. We have that
\begin{align*}
	\phi(ab) &= (ab)^{n} \\
	&= b^{n}a^{n} \\
	&= a^{n}b^n \because{\text{G is abelian}} \\
	&= \phi(a)\phi(b)
\end{align*}

Hence, $\phi : G \mapsto G$ is indeed a group homomorphism. \qed \\

10) If $\phi : G \to H$ is a group homomorphism and $G$ is cyclic, prove that $\phi(G)$ is also cyclic.

\underline{Proof}: Let $G = \langle g \rangle$. We want to show there exist $x \in \phi(G)$ such that $\phi(G) = \langle x \rangle$. It is natural to think $\phi(g)$ generates $\phi(G)$. Then we have that
$$\phi(g^n) = \phi\underbrace{(g \cdot g \cdot g \cdots g)}_{\text{n times}} = \underbrace{\phi(g)\phi(g) \cdots \phi(g)}_{\text{n times}} = \big(\phi(g)\big)^n.$$

Since every element of $G$ is of the form $g^{n}$, then this shows that every element $\phi(G)$ is of the form $\big(\phi(g)\big)^n$. Thus $\phi(G)$ is also cyclic and $\phi(G) = \langle \phi(g) \rangle$. \qed \\

\pagebreak 8) Prove that $\mathbb{Q}$ is not isomorphic to $\mathbb{Z}$. 

\underline{Proof}: Without loss of generality, assume the group binary operation is $+$ for both groups. We claim that $\mathbb{Q} \not \cong \mathbb{Z}$  since $\mathbb{Z}$ is cyclic ($\mathbb{Z} = \langle 1 \rangle$) while $\mathbb{Q}$ is not. If $\mathbb{Q}$ were isomorphic to $\mathbb{Z}$ then $\mathbb{Q}$ must be cyclic. We will show that $\mathbb{Q}$ is not cyclic. 

Clearly $\mathbb{Q}$ is not generated by $0$ since $\langle 0 \rangle = \{ 0 \}$. So we will consider any cyclic subgroup of $\mathbb{Q}$ generated by a non-zero element $a \in \mathbb{Q}$. Then, by definition of being a cyclic group, we have that 
$$Q = \langle a \rangle = \{ na \, \vert\, n \in \mathbb{Z} \}.$$
Notice that this subgroup in particular does not contain the element $\frac{a}{2}$ for example. If it did, then we would have that $\frac{a}{2} = \frac{1}{2}a = na$ for some $n \in \mathbb{Z}$. But, earlier we showed that $a \neq 0$ since $\mathbb{Q}$ is not generated by $0$. Thus, $n$ must be $\frac{1}{2}$. However, $\frac{1}{2} \not \in \mathbb{Z}$. Therefore, $\mathbb{Q}$ is not cyclic.

Since we have shown that $\mathbb{Q}$ is not cyclic while $\mathbb{Z}$ is, it follows that $\mathbb{Q} \not \cong \mathbb{Z}$ by the Structure Theorem for groups presented in recitation. \qed \\

47) If $G \cong \overline{G}$ and $H \cong \overline{H}$, show that $G \times H \cong \overline{G} \times \overline{H}.$

\underline{Proof}: To show two groups are isomorphic, we need to define a bijective group homomorphism between the two groups.  Since $G \cong \overline{G}$, there exist a bijective group homomorphism $\phi : G \mapsto \overline{G}$. Similarly, since $H \cong \overline{H}$, there exist a bijective group homomorphism $\psi : H \mapsto \overline{H}$. So define 
$$\phi \times \psi : G \times H \mapsto \overline{G} \times \overline{H}$$
by $(\phi \times \psi)(g, h) = (\phi(g), \psi(h))$ where $G \times H := \{ (g,h) \ \vert\, g \in G, h \in H \}$. We now need to check that $\phi \times \psi$ is indeed a bijective group homomorphism.

i) To be explicit, we should verify that $\phi \times \psi$ is well-defined. It is well-defined if $(g, h) = (g', h') \implies (\phi \times \psi)(g, h) = (\phi \times \psi)(g', h').$ By definition of $\phi \times \psi$ to the left hand side, we have
$$(\phi(g), \psi(h)) = (\phi(g'), \psi(h'))$$
which is equivalent to the right hand side, so $\phi \times \psi$ is well-defined.

ii) We now show $\phi \times \psi$ is one-to-one. Let $(g_1, h_1), (g_2, h_2) \in G \times H$ such that 
$$(\phi \times \psi)(g_1, h_1) = (\phi \times \psi)(g_2, h_2).$$
Then we have that $(\phi(g_1), \psi(h_1)) = (\phi(g_2), \psi(h_2))$ which is true if and only if $\phi(g_1) = \phi(g_2)$ and $\psi(h_1) = \psi(h_2)$. Since both $\phi$ and $\psi$ are bijective, we have that $g_1 = g_2$ and $h_1 = h_2$ respectively. Thus $(g_1, h_1) = (g_2, h_2)$ and $\phi \times \psi$ is one-to-one.

iii) We now show that $\phi \times \psi$ is onto and thus we will have shown it is bijective. Let $(\overline{g}, \overline{h}) \in \overline{G} \times \overline{H}$. We want to show there exist $(g, h)$ such that
$$(\phi \times \psi)(g, h) = (\overline{g}, \overline{h}).$$
We have that
\begin{align*}
	(\phi \times \psi)(g, h) &= \big(\phi(g), \psi(h) \big) \\
	&= (\overline{g}, \overline{h}).
\end{align*}
So $\phi \times \psi$ is onto and we then have that it is bijective.

iv) Lastly we need to show $\phi \times \psi$ is a homomorphism. Let $(g_1, h_1), (g_2, h_2) \in G \times H$. We want to show
\begin{equation}
(\phi \times \psi)\big((g_1, h_1)(g_2, h_2)\big) = (\phi \times \psi)(g_1, h_1) (\phi \times \psi)(g_2, h_2)
\end{equation}
By definition, we have that $(g_1h_1)(g_2h_2) = (g_1g_2, h_1h_2)$. Working with the left hand side of equation (1), we have that
\begin{align*}
	(\phi \times \psi)\big((g_1, h_1)(g_2, h_2)\big) &= (\phi \times \psi)(g_1g_2, h_1h_2) \\
	&= \big(\phi(g_1g_2), \psi(h_1h_2)\big) \\
	&= \big(\phi(g_1)\phi(g_2), \psi(h_1)\psi(h_2)\big) \because \phi, \psi \, \text{homomorphisms}
\end{align*}

Working with the right hand side of equation (1), we also see that
\begin{align*}
	(\phi \times \psi)(g_1, h_1) (\phi \times \psi)(g_2, h_2) &= \big(\phi(g_1), \psi(h_1)\big)\big(\phi(g_2), \psi(h_2)\big) \\
	&= \big(\phi(g_1)\phi(g_2), \psi(h_1)\psi(h_2)\big)
\end{align*}

Thus, $\phi \times \psi$ is indeed a group homomorphism. Since the function $\phi \times \psi$ is bijective and a group homomorphism, we have that $$G \times H \cong \overline{G} \times \overline{H}.$$ \qed \\

4) Let $T$ be the group of nonsingular upper triangular $2 \times 2$ matrices with entries in $\mathbb{R}$. Let $U$ consist of matrices of the form 
$$\begin{pmatrix}
	1 & x \\
	0 & 1
\end{pmatrix}$$
where $x \in \mathbb{R}$.

4a) Show that $U$ is a subgroup of $T$.

\underline{Proof}: To show $U < T$ we will use the Subgroup Test.

i) We need to show the identity of $T$, $I_2$ is an element of $U$. Simply let $x = 0$ and we get the identity matrix. So the identity $I_2 \in U$.

ii) If $h_1, h_2 \in U$, we need to show $h_1h_2 \in U$. Let $h_1 =
\begin{pmatrix}
	1 & x \\ 
	0 & 1
\end{pmatrix}$ and $h_2 = 
\begin{pmatrix}
	1 & y \\ 
	0 & 1
\end{pmatrix}$ for $x, y \in \mathbb{R}$. Then we have that
$$h_1h_2 = 
\begin{pmatrix}
	1 & y + x \\
	0 & 1
\end{pmatrix}$$ which is an element of $U$ since $y + x \in \mathbb{R}$.

iii) Lastly, we need to show that if $h \in U$ then $h^{-1} \in U$. Let $h =
\begin{pmatrix}
	1 & x \\ 
	0 & 1
\end{pmatrix}.$ Then $h^{-1} =
\begin{pmatrix}
	1 & -x \\ 
	0 & 1
\end{pmatrix}.$ Since $x \in \mathbb{R}, -x \in \mathbb{R}$. Thus $h^{-1} \in U.$ \qed \\

4b) Prove that $U$ is abelian. 

\underline{Proof}: To show $U$ is abelian, we will simply calculate $h_1h_2$ and compare it to $h_2h_1$ for $h_1, h_2 \in U$. If $h_1h_2 = h_2h_1$ then $U$ is abelian.

Let $h_1 = 
\begin{pmatrix}
	1 & x \\ 
	0 & 1
\end{pmatrix}$ and 
$h_2 =
\begin{pmatrix}
	1 & y \\ 
	0 & 1
\end{pmatrix}$. Then we have that 
$$h_1h_2 = 
\begin{pmatrix}
	1 & y + x \\
	0 & 1		
\end{pmatrix}$$
and also that
$$h_2h_1 = 
\begin{pmatrix}
	1 & x + y \\
	0 & 1
\end{pmatrix}.$$ Since $x, y \in \mathbb{R}, x + y = y + x.$ Thus, $h_1h_2 = h_2h_1$ and so $U$ is abelian. \qed \\

4c) Prove that $U$ is normal in $T$. 

\underline{Proof}: To show $U$ is normal in $T$, we need to show that for all $t \in T$, $tUt^{-1} \subset U$. 

Let $t = 
\begin{pmatrix}
	a & b \\
	0 & c
\end{pmatrix}$ for $a, b, c \in \mathbb{R}$. Then 
\begin{align*}
	t^{-1} &=
	\begin{pmatrix}
		c & -b \\
		0 & a
	\end{pmatrix} \\
	&=
	\begin{pmatrix}
		\frac{1}{a} & -\frac{b}{ac} \\[6pt]
		0 & \frac{1}{c}
	\end{pmatrix}	
\end{align*}
So we then have that 
\begin{align*}
	tUt^{-1} &=
	\begin{pmatrix}
		a & c \\
		0 & c
	\end{pmatrix}
	\begin{pmatrix}
		1 & x \\
		0 & 1
	\end{pmatrix}
	\begin{pmatrix}
		\frac{1}{a} & -\frac{b}{ac} \\[6pt]
		0 & \frac{1}{c}
	\end{pmatrix}\\
	&=
	\begin{pmatrix}
		a & ax + b \\
		0 & c
	\end{pmatrix}
	\begin{pmatrix}
		\frac{1}{1} & -\frac{b}{ac} \\[6pt]
		0 & \frac{1}{c}
	\end{pmatrix} \\
	&=
	\begin{pmatrix}
		1 & -\frac{b}{c} + \frac{ax+b}{c} \\[6pt]
		0 & 1
	\end{pmatrix} \\	
	&=
	\begin{pmatrix}
		1 & \frac{ax}{c} \\[6pt]
		0 & 1
	\end{pmatrix}	
\end{align*}
Notice that $\frac{ax}{c} \in \mathbb{R}$ since $a, x, c \in \mathbb{R}$ and note that since $ac \neq 0$, then $c \neq 0$ so we are not dividing by $0$. Thus, $tUt^{-1} \subset U.$ and hence $U$ is normal in $T$. \qed \\

\pagebreak 4e) Is $T$ normal in $GL_{2}(\mathbb{R})$?

\underline{Proof}: No, $T$ is \textit{not} normal in $GL_{2}(\mathbb{R})$. Consider the following matrices:
$$t =
\begin{pmatrix}
	1 & 1 \\
	0 & 1
\end{pmatrix} \in T \text{\, and \,} g = 
\begin{pmatrix}
	1 & 0 \\
	1 & 1
\end{pmatrix} \in GL_{2}(\mathbb{R}).$$ Then we have that
\begin{align*}
	gtg^{-1} &=
	\begin{pmatrix}
		1 & 0 \\
		1 & 1
	\end{pmatrix}
	\begin{pmatrix}
		1 & 1 \\
		0 & 1
	\end{pmatrix}
	\begin{pmatrix}
		1 & 0 \\
		-1 & 1
	\end{pmatrix} \\
	&=
	\begin{pmatrix}
		1 & 1 \\
		1 & 2
	\end{pmatrix}
	\begin{pmatrix}
		1 & 0 \\
		-1 & 1
	\end{pmatrix} \\
	&=
	\begin{pmatrix}
		0 & 1 \\
		-1 & 2
	\end{pmatrix}
\end{align*}

Hence we see that $gtg^{-1} \not \in T.$ Then $gTg^{-1} \not \subset T.$ Therefore, $T$ is not a normal subgroup of $GL_2(\mathbb{R})$. \qed \\

\end{document}