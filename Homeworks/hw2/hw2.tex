\documentclass{article}
\usepackage{times, amsfonts}
\usepackage{amsmath, amsthm, amssymb}

\title{Abstract Algebra Homework 2}
\author{\textit{Joe Loser}}
\date{February 1, 2016}

\begin{document}
\maketitle

\noindent This problem set includes problems from sections 2.3, 3.4, and two extra problems from lecture.  \\

\noindent 31) Using the fact that $2$ is prime, show that there do not exist integers $p$ and $q$ such that $p^{2} = 2q^{2}$. \\

\noindent
\underline{Proof}: We will prove the statement using contradiction. First, find the largest number $k \in \mathbb{N}$ such that $2^{k}$ divides divides both $p$ and $q$.  It should be easy to see that $k$ will be $0$ if either $p$ or $q$ is odd. \\

\noindent
Now, let $a := \frac{p}{2^{k}}$ and $b := \frac{q}{2^{k}}$. Since we found the \textit{largest} $k$, we are certain that either $a$ or $b$ is odd.  Otherwise, we could have increased $k$. By our definitions of $a$ and $b$, $p = a 2^{k}$ and $q = b 2^{k}$. Then,

\begin{align*}
p^{2} &= 2q^{2} \\
(a 2^{k})^{2} &= 2(b 2^{k})^{2} \\
a^{2} (2^{k})^{2} &= 2b^{2}(2^{k})^2 \\
a^{2} &= 2b^{2}
\end{align*}

\noindent
Thus, we see that $a^{2}$ is even.  Using the given fact that $2$ is prime, $a$ must also be even. In fact, $b$ must also be odd due to our previous work. Let $c := \frac{a}{2}$. So $a = 2c$. Then,
\begin{align*}
(2c)^{2} &= 2b^{2} \\
4c^{2} &= 2b^{2} \\
2c^{2} &= b^{2}
\end{align*}

\noindent
But now we have shown that $b^{2}$ is even which implies that $b$ is even since $2$ is a prime. However, this contradicts how we constructed such an $a$ and $b$. By contradiction, no such integers $p$ and $q$ can exist that satisfy the condition $p^{2} = 2q^{2}$. It is well know that $\sqrt{2} \not \in \mathbb{Q}$. \qed \\ \\ \\

\noindent 1b) Find all $x \in \mathbb{Z}$ such that $5x + 1 \equiv 13\ (\bmod\ 23) $. \\

\noindent
\underline{Proof}: We will begin with the definition of congruent mod $m$, find the least residue class, and then find the other solutions that satisfy the equation. \\

\noindent
Let $a$ and $b$ be integers and $m$ be a natural number. Then, $a$ is congruent to $b$ modulo $m$, i.e. $a \equiv b\ (\bmod\ m)$ if $m \rvert (a-b)$ by definition. In our case, $a = 5x + 1, b = 13,$ and $m = 23$. So, we want to find all $x \in \mathbb{Z}$ such that $23 \rvert (5x-12)$. Beginning with $x = 0$ and going up to $x = 23$, we see that $x = 7$ is a solution and is indeed the least residue class. If $x = 7, 5x-12 = 23$. We can keep on generating more $x$ values that satisfy the equation by adding the modulus value $m = 23$ to our latest $x$ value. So, all $x \in \mathbb{Z}$ that work are $x = 7 + 23n$ for $n \in \mathbb{Z}$. \qed \\

\noindent
Remark: I exploited the fact that $23$ is prime which is what allows me to find the \textit{unique} least residue class technically. While we have not proved this, it is perfectly acceptable and I did indeed brute force all the way from $x = 0$ to $x = 23$ to find all $x$ that satisfy the equation initially before just adding $\bmod\ 23$ to generate more possible values $x$ can take on.  The alternative, more systematic approach is to use the Euclidean Algorithm rather than just working by inspection or brute force as this would not be ideal if $m$ is large. \\

\noindent 7) Let $S = \mathbb{R} \setminus \{-1\}$ and define a binary operation on $S$ by $a * b = a + b + ab.$ Prove that $(S, *)$ is an abelian group. \\

\noindent
\underline{Proof}: First, we will prove that the set $S$ is indeed a group and then we will show that the group $S$ is in fact abelian. \\

\noindent
To show that the set $S$ is a group with a binary operation $*$, it needs to satisfy the following conditions:
\begin{enumerate}
  \item $*$ is associative, i.e. $(a * b) * c = a * (b * c)$ for $a, b, c \in S$.
  \item $S$ is closed under $*$.
  \item There exists an element $e \in S$ such that for any element $a \in S, e * a = a * e = a$ (identity property)
  \item For each element $a \in S$, there exists an inverse element in $S$ which we denote $a^{-1}$ such that $a * a^{-1} = a^{-1} * a = e$ (inverse property)
  \item * is a binary operator
\end{enumerate}

\noindent
We show each of these conditions in order now.
\begin{enumerate}
	\item We have to show $(a * b) * c = a * (b * c)$ for $a, b, c \in S$. By applying the definition of our binary operation $*$ we have that
	\begin{align*}
		(a * b) * c &= (a + b + ab) * c	\\
		&= (a + b + ab) + c + (a + b + ab) c \\
		&= a + b + ab + c + ac + bc + abc \\
		&= a + b + c + bc + a(b + c + bc) \\
		&= a + (b + c + bc) + a(b + c + bc) \\		
		&= a * (b * c)
	\end{align*}
	
	\item We need to show that $S$ is closed under $*$, i.e. we need to make sure $a * b$ never equals $-1$.  Note that if $a * b = a + b + ab = -1$, then $a + b + ab + 1 = 0$.  This is equivalent to $(b + 1) + a(1 + b) = 0$ by factoring out an $a$ where we can. This is the same as $(b+1)(a+1) = 0$. Now, note that $a$ and $b$ cannot be $-1$ because the set $S = \mathbb{R} \setminus \{-1\}$ and $a$ and $b$ $\in S$. Since $a$ and $b$ are not $-1$, neither is $a + b + ab$. Therefore, $S$ is closed under the binary operation $*$.
	
	\item We need to show that $S$ has an identity element. Notice that $a * 0 = a + 0 + 0 = a$, so $S$ does indeed have an identity element and it is $0$. One can also see that $a * 0 = 0 * a$ since $S$ is abelian.
	
	\item We need to show that every $a \in S$ has an inverse element. Choose any $a \in S$. We want to find an inverse for $a$, so we want to find a $b \neq -1$ such that $a * b = 0$. One can also see this is equal to $b * a$ since $S$ is abelian.
	\begin{center}
		$ a * b = 0$ \\
		$ ab + a + b = 0$ \\
		$ b(a + 1) = -a$ \\
		$ b = -\dfrac{a}{a+1} $
	\end{center}	
\noindent
So, let $b = \frac{-a}{1+a}$. Now note that $b$ cannot be $-1$, otherwise we would have that $-1 = \frac{-a}{1+a} \implies a = 1 + a$ which is not possible. As a result, $b \in S$ and $a \neq -1$. Then, 
	\begin{align*}
		a * b &= a + \dfrac{-a}{1+a} + a\dfrac{-a}{1+a} \\
		&= \dfrac{a(1+a)}{1+a} + \dfrac{-a}{1+a} + \dfrac{-a^{2}}{1+a} \\
		&= 0
	\end{align*}
which is the identity.
\item Lastly, to show $*$ is a binary operation, if $a, b \neq -1$, we need to show $a * b \neq -1$. The only way $a * b = -1$ is if either $a$ or $b$ is $-1$. Thus $a * b \neq -1$ for all $a, b \in S$ and we have that $*$ is a binary operation.
\end{enumerate}

\noindent
So, we have shown that $S$ is indeed a group.  Now, in order for $S$ to be an abelian group, $S$ needs to also have the property that $a * b  = b * a$ for all $a, b \in S$. Notice that $a * b = a + b + ab = b + a + ab = b * a$ for all $a, b \in S$. So, the binary operation $*$ is commutative. \\

\noindent
Therefore, the set $S = \mathbb{R} \setminus \{-1\}$ is an abelian group. \qed \\

\noindent E1) Let $a, b, g, s \in \mathbb{Z}$. If $b \neq 0$ and $a = bg + s$, show that $gcd(a, b) = gcd(b, s)$. \\

\noindent
\underline{Proof}: We will show that if $a = bg + s$ then there is an integer $d$ that is a common divisor of $a$ and $b$ if and only if $d$ is a common divisor of $b$ and $s$. \\

\noindent
Let $d$ be a common divisor of $a$ and $b$.  By definition of common divisor, $d \rvert a$ and $d \rvert b$. Hence, by Corollary A6, $d \rvert(a - bg)$ which means $d \rvert s$ since $s = a - bg$. Thus $d$ is a common divisor of $b$ and $s$. \\

\noindent
Now, suppose that $d$ is a common divisor of $b$ and $s$. By definition of common divisor, $d \rvert b$ and $d \rvert s$. Hence, by Corollary A6, $d \rvert (bg + s)$ so $d \rvert a$ since $a = bg + s$. Thus, $d$ must be a common divisor of $a$ and $b$. \\

\noindent
Thus, the set of common divisors of $a$ and $b$ are the same as the set of common divisors of $b$ and $s$. It follows that $d$ is the \textit{greatest} common divisor of $a$ and $b$ if and only if $d$ is the greatest common divisor of $b$ and $s$. \qed \\

\noindent E2) Show that the set $S$ formed in the proof of Theorem $A5$ consists precisely of all the positive multiples of $d = gcd(a, b)$. \\

\noindent
To refresh ourselves, Theorem $A5$ is provided below for convenience. Let $a, b \in \mathbb{Z}$ with at least one of which is non-zero. Then,
\begin{enumerate}
	\item $gcd(a, b)$ exists and is unique
	\item There exists $r, s \in \mathbb{Z}$ such that $gcd(a, b) = ra + sb$.
\end{enumerate}
In the proof of this theorem, we let $S := \{am + bn \ \rvert \ m, n \in \mathbb{Z} \,\} \cap \mathbb{Z}_{\geq 1}$ and we also showed that $d = gcd(a, b)$ was the smallest element of $S$. We want to show that this set $S$ contains all of the positive multiples of $d = gcd(a, b)$. \\

\noindent
\underline{Proof}: 
First, let $c \in S, d = gcd(a, b)$. Then, by our definition of the set S, there exist $m, n \in \mathbb{Z}$ such that $c = ma + nb$. Since $d = gcd(a, b)$, $d \rvert a$ and $d \rvert b$. So, there exist integers $x$ and $y$ such that $a = xd$ and $b = yd$. Then,
\begin{align*}
c &= ma + nb \\
&= mxd + nyd \\
&= (mx + ny)d
\end{align*}
Hence, $c = kd$ where $k = mx + ny$ and we conclude that $d \rvert c$. \\

\noindent
Showing the opposite direction now, again, let $c$ be an integer and assume that $d \rvert c$. Then there exist an integer $x'$ such that $c = x'd$. By Theorem A5, then there exist integers $r$ and $s$ such that $d = ar + bs$. Substituting this value for $d$, we then have that
\begin{align*}
c &= x'd \\
&= x'(ar + bs) \\
&= x'ar + x'bs \\
&= a(rx') + b(sx')
\end{align*}
Hence, $c = am + bn$ where $m = rx'$ and $n = sx'$. \\

\noindent
Therefore, the set $S$ consists of all of the positive multiples of $gcd(a, b)$. \qed \\

\end{document}
