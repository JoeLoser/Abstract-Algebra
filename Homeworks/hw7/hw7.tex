\documentclass{article}
\usepackage{amsmath, amsthm, amssymb}
\usepackage{parskip} % so I don't have to use \noindent everywhere
\usepackage{mathtools}
\usepackage[none]{hyphenat} % don't split up words that flow onto next line
\usepackage{microtype,todonotes}
\usepackage[a4paper,text={16.5cm,25.2cm},centering]{geometry}
\usepackage[compact,small]{titlesec}
\setlength{\parskip}{1.2ex}
\setlength{\parindent}{0em}

% To prevent orphans and widows (https://en.wikipedia.org/wiki/Widows_and_orphans)
\clubpenalty = 10000
\widowpenalty = 10000
\usepackage{kpfonts}
\usepackage[T1]{fontenc}

\title{Abstract Algebra Homework 7}
\author{\textit{Joe Loser}}
\date{March 18, 2016}
\begin{document}
\maketitle

This problem set includes problems $10.3$ numbers $4d), 11.3$ numbers $7, 16,17$ and $11.4$ number 5.

4) Let $T$ be the group of nonsingular upper triangular $2 \times 2$ matrices with entries in $\mathbb{R}$. Let $U$ consist of matrices of the form 
$$\begin{pmatrix}
1 & x \\
0 & 1
\end{pmatrix}$$
where $x \in \mathbb{R}$. \\

4d) Show that $T/U$ is abelian. \\

\underline{Proof}: Note that we have already showed that $U$ is normal in $T$ in part 4c). To show that $T/U$ is abelian, we need to show that $(AU)(BU) = (BU)(AU)$ for all $A, B \in T$.

Let $A =
\begin{pmatrix}
	a & b \\ 
	0 & c
\end{pmatrix}$ and let $B =
\begin{pmatrix}
	a' & b' \\
	0 & c'
\end{pmatrix}$. Then we have that $$AB = 
\begin{pmatrix}
	aa' & ab' + bc' \\
	0 & cc'
\end{pmatrix}$$ and $$BA = 
\begin{pmatrix}
	a'a & a'b + b'c \\
	0 & c'c
\end{pmatrix}.$$ 

This shows that $AB \neq BA$ in general. However, we want to show that $(AU)(BU) = (BU)(AU)$. Note that $(AU)(BU) = ABU$ and $(BU)(AU) = BAU$ since $U$ is normal. Let $C =
\begin{pmatrix}
	1 & z \\
	0 & 1
\end{pmatrix} \in U$ where $z \in \mathbb{R}$. Then we have that
$$ABU =
\begin{pmatrix}
	aa' & z(ab' + bc') \\
	0 & cc'
\end{pmatrix}$$ and 
$$BAU = 
\begin{pmatrix}
a'a & z(a'b + b'c) \\
0 & c'c
\end{pmatrix}.$$ 

Notice that $aa' = a'a$ and $cc' = c'c$ since $a, a', c, c' \in \mathbb{R}$. So we see that $ABU$ and $BAU$ only differ in the upper right entry. This does not matter though since $U$ is matrices of the form 
$$\begin{pmatrix}
1 & x \\
0 & 1
\end{pmatrix}$$ where $x \in \mathbb{R}$. Notice that both $z(ab' + bc')$ and $z(a'b + b'c) \in \mathbb{R}$. Thus, $AB$ and $BA$ define the same coset in $U$, meaning that $ABU = BAU$. Thus $T/U$ is abelian. \qed \\

7) In the group $\mathbb{Z}_{24}$, let $H = \langle 4 \rangle$ and $N = \langle 6 \rangle.$\\

a) List the elements in $H + N$ and $H \cap N$.\\

b) List the cosets in $HN/N$, showing the elements in each coset.\\

c) List the costs in $H/(H \cap N)$, showing the elements in each coset.\\

d) Give the correspondence between $HN/N$ and $H/(H \cap N)$ described in the proof of the Second Isomorphism Theorem. \\

16) If $H$ and $K$ are normal subgroups of $G$ and $H \cap K = \{ {e} \}$, prove that $G$ is isomorphic to a subgroup of $G/H \times G/K$. \\

17) Let $\phi : G_1 \mapsto G_2$ be a surjective group homomorphism. Let $H_1$ be a normal subgroup of $G_1$ and suppose that $\phi(H_1) = H_2$. Prove or disprove that $G_1/H_1 \cong G_2/H_2$. \\

\underline{Proof}: We will disprove that $G_1/H_1 \cong G_2/H_2$ by giving a counterexample. \todo{Give counter example.} \qed \\

5) Let $G$ be a group and let $i_g$ be an inner automorphism of $G$ and define a map $G \mapsto Aut(G)$ by $g \mapsto i_g$. Prove that this map is a homomorphism with image $Inn(G)$ and kernel $Z(G)$. Use this result to conclude that 
$$G/Z(G) \cong Inn(G)$$




\end{document}