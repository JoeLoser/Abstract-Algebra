\documentclass{article}
\usepackage{amsmath, amsthm, amssymb}
\usepackage{parskip} % so I don't have to use \noindent everywhere
\usepackage{mathtools}
\usepackage[none]{hyphenat} % don't split up words that flow onto next line
\usepackage{microtype,todonotes}
\usepackage[a4paper,text={16.5cm,25.2cm},centering]{geometry}
\usepackage[compact,small]{titlesec}
\setlength{\parskip}{1.2ex}
\setlength{\parindent}{0em}
\DeclareMathOperator{\lcm}{lcm}

% To prevent orphans and widows (https://en.wikipedia.org/wiki/Widows_and_orphans)
\clubpenalty = 10000
\widowpenalty = 10000
\usepackage{kpfonts}
\usepackage[T1]{fontenc}

\title{Abstract Algebra Homework 7}
\author{\textit{Joe Loser}}
\date{March 18, 2016}
\begin{document}
\maketitle

This problem set includes problems $10.3$ numbers $4d), 11.3$ numbers $7, 16,17$ and $11.4$ number 5.

4) Let $T$ be the group of nonsingular upper triangular $2 \times 2$ matrices with entries in $\mathbb{R}$. Let $U$ consist of matrices of the form 
$$\begin{pmatrix}
1 & x \\
0 & 1
\end{pmatrix}$$
where $x \in \mathbb{R}$. \\

4d) Show that $T/U$ is abelian. \\

\underline{Proof}: Note that we have already showed that $U$ is normal in $T$ in part 4c). To show that $T/U$ is abelian, we need to show that $(AU)(BU) = (BU)(AU)$ for all $A, B \in T$.

Let $A =
\begin{pmatrix}
	a & b \\ 
	0 & c
\end{pmatrix}$ and let $B =
\begin{pmatrix}
	a' & b' \\
	0 & c'
\end{pmatrix}$. Then we have that $$AB = 
\begin{pmatrix}
	aa' & ab' + bc' \\
	0 & cc'
\end{pmatrix}$$ and $$BA = 
\begin{pmatrix}
	a'a & a'b + b'c \\
	0 & c'c
\end{pmatrix}.$$ 

This shows that $AB \neq BA$ in general. However, we want to show that $(AU)(BU) = (BU)(AU)$. Note that $(AU)(BU) = ABU$ and $(BU)(AU) = BAU$ since $U$ is normal. Let $C =
\begin{pmatrix}
	1 & z \\
	0 & 1
\end{pmatrix} \in U$ where $z \in \mathbb{R}$. Then we have that
$$ABU =
\begin{pmatrix}
	aa' & z(ab' + bc') \\
	0 & cc'
\end{pmatrix}$$ and 
$$BAU = 
\begin{pmatrix}
a'a & z(a'b + b'c) \\
0 & c'c
\end{pmatrix}.$$ 

Notice that $aa' = a'a$ and $cc' = c'c$ since $a, a', c, c' \in \mathbb{R}$. So we see that $ABU$ and $BAU$ only differ in the upper right entry. This does not matter though since $U$ is matrices of the form 
$$\begin{pmatrix}
1 & x \\
0 & 1
\end{pmatrix}$$ where $x \in \mathbb{R}$. Notice that both $z(ab' + bc')$ and $z(a'b + b'c) \in \mathbb{R}$. Thus, $AB$ and $BA$ define the same coset in $U$, meaning that $ABU = BAU$. Thus $T/U$ is abelian. \qed \pagebreak

7) In the group $\mathbb{Z}_{24}$, let $H = \langle 4 \rangle$ and $N = \langle 6 \rangle.$\\

a) List the elements in $H + N$ and $H \cap N$.\\

\underline{Solution}: We have that
\begin{align*}
	H+N &= \{ h + n \ \vert \ h \in H \text{and} \, n \in N\} \\
	&= \{ h + n \ \vert \ h \in \langle 4 \rangle \, \text{and} \, n \in \langle 6 \rangle\} \\
	&= \{ 0, 2, 4, 6, 8, 10, 12, 14, 16, 18, 20, 22 \} \\
	&= \langle 2 \rangle
\end{align*}

We also see that $H \cap N = \{ 0, 12 \}.$

Remarks: Note that $H + N = \big\langle \gcd(4,6) \big\rangle $ in $\mathbb{Z}_{24}$ and also that $H \cap N = \big\langle \lcm(4, 6) \big\rangle \mathbb{Z}$ in $\mathbb{Z}_{24}$. \qed \\

b) List the cosets in $HN/N$, showing the elements in each coset.\\

\underline{Solution}: We know that 
$$ (H + N) / N := \{ g + n \, \vert \, g \in H + N\}. $$

So we see that the cosets that partition $H+N$ are the following:
\begin{align*}
	0 + N &= \{ 0, 6, 12, 18\} \\
	2 + N &= \{ 2, 8, 14, 20 \} \\
	4 + N &= \{ 4, 10, 16, 22\}
\end{align*}

c) List the costs in $H/(H \cap N)$, showing the elements in each coset.\\

\underline{Solution}: We know that
$$ H / (H \cap N) := \{ aH \cap N \, \vert \, a \in H \}.$$

So we see that the cosets that partition the group $H$ are the following:
\begin{align*}
	0 + H \cap N &= \{0, 12\} \\
	4 + H \cap N &= \{4, 16\}  \\
	8 + H \cap N &= \{8, 20\}
\end{align*}

d) Give the correspondence between $()H+N)/N$ and $H/(H \cap N)$ described in the proof of the Second Isomorphism Theorem. \\

\underline{Solution}: Recall that all subgroups of an abelian group are normal. Since $\mathbb{Z}_{24}$ is an abelian group, we have that $H$ and $N$ are normal in $\mathbb{Z}_{24}$. So we can apply the Second Isomorphism Theorem which tells us
$$H / (H \cap N) \cong (H+N) / N.$$ \qed \\

16) If $H$ and $K$ are normal subgroups of $G$ and $H \cap K = \{ {e} \}$, prove that $G$ is isomorphic to a subgroup of $G/H \times G/K$. \\

\underline{Proof}: To show that $G$ is isomorphic to a subgroup of $G/H \times G/K$, we need to define a function $\phi$ and show it is a group homomorphism. Then we will show that $\ker \phi = \{e\}$ and use the First Isomorphism Theorem.

Let $\phi : G \mapsto G/H \times G/K$ be defined as $\phi(g) = (gH, gK)$. Clearly the function $\phi$ is well-defined. We now need to show this is indeed a group homomorphism. Let $a, b \in G$. Then
\begin{align*}
	\phi(ab) &= (abH, abK) \\
	&= (aH, aK)(bH, bK) \\
	&= \phi(a)\phi(b).
\end{align*} So $\phi$ is a homomorphism. From the First Isomorphism Theorem, we know that 
$$G / \ker \phi \cong \phi(G).$$

Since $\phi(G)$ is a subgroup of $G/H \times G/K$ and we want to show $G \cong \phi(G)$, it suffices for us to show that $\ker \phi = \{ e \}$, i.e. $\ker \phi = H \cap K$. 

Let $g \in \ker \phi$. Then $\phi(g) = (gH, gK) = (H, K)$. That is, $gH = H$ and $gK = K$. Thus $g \in H \cap K$ and we have that $\ker \phi \subset H \cap K$. Contrarily, if $g \in H \cap K$ then we clearly see that $\phi(g) = (gH, gK) = (H, K)$ so that $H \cap K \subset \ker \phi$. Thus $\ker \phi = H \cap K$.

Since we have shown that $\ker \phi = H \cap K$, i.e. $\phi$ is one-to-one, we have proven that 
$$G \cong \phi(G)$$
which is a subgroup of $G/H \times G/K$. \qed \\

17) Let $\phi : G_1 \mapsto G_2$ be a surjective group homomorphism. Let $H_1$ be a normal subgroup of $G_1$ and suppose that $\phi(H_1) = H_2$. Prove or disprove that $G_1/H_1 \cong G_2/H_2$. \\

\underline{Proof}: We will disprove that $G_1/H_1 \cong G_2/H_2$ by giving a counterexample.

Let $G_1 = \mathbb{Z}, G_2 = \mathbb{Z}, H_1 = 2\mathbb{Z}, H_2 = 3\mathbb{Z}$. Then we see that obviously $G_1 \cong G_2$ and $H_1 \cong \mathbb{Z} \cong H_2$. However, for the quotient groups, we have that 
$$G_1 / H_1 = \mathbb{Z} / 2\mathbb{Z} \cong \mathbb{Z}_2$$
and
$$G_2 / H_2 = \mathbb{Z} / 3\mathbb{Z} \cong Z_3.$$

So $G_1 / H_1 \not \cong G_2 / H_2$. We have not explicitly defined our surjective group homomorphism $\phi$ yet. 

All possible homomorphisms from $\mathbb{Z} \mapsto \mathbb{Z}$ are determined by the image of the generator, i.e. $1$. So suppose then that $\phi(1) = n$ for $n \in \mathbb{Z}$. Then the homomorphism is
\begin{align*}
	\phi(k) &= \phi(k \cdot 1) \\
	&= k \phi(1) \\
	&= kn
\end{align*}
for all $k \in \mathbb{Z}$. Thus, \textit{all} of the possible homomorphisms from $\mathbb{Z} \mapsto \mathbb{Z}$ are of the form $\phi_n : \mathbb{Z} \mapsto \mathbb{Z}$ defined by $\phi_n(k) = kn$ for all $k, n \in \mathbb{Z}$. However, for the purposes of this problem, we are only interested in homomorphisms of $\mathbb{Z}$ onto $\mathbb{Z}$. Since $\mathbb{Z}$ only has two generators: $1$ and $-1$, we have that the two possible onto homomorphisms are
$$\phi_1 : \mathbb{Z} \mapsto \mathbb{Z}; \phi_1(k) = k$$
$$\phi_{-1} : \mathbb{Z} \mapsto \mathbb{Z}; \phi_{-1}(k) = -k.$$ 

Since we have constructed a specific example showing that $G_1/H_1 \cong G_2/H_2$ does not always hold, we are done. \qed \\\

5) Let $G$ be a group and let $i_g$ be an inner automorphism of $G$ and define a map $G \mapsto Aut(G)$ by $g \mapsto i_g$. Prove that this map is a homomorphism with image $Inn(G)$ and kernel $Z(G)$. Use this result to conclude that 
$$G/Z(G) \cong Inn(G)$$

\underline{Proof}: Recall that $i_g(x) := gxg^{-1}$. We first show that the map is a homomorphism. Let $a, b \in G$. Then
\begin{align*}
	i_{ab}(x) &= (ab)x(ab)^{-1} \\
	&= abxb^{-1}a^{-1} \\
	&= a(bxb^{-1})a^{-1} \\
	&= i_a(i_b(x)) \\
	&= (i_ai_b)(x)
\end{align*}
Since $i_{ab} = i_ai_b$ our map is a homomorphism.

By definition, we know that $Inn(G)$ is the set of inner automorphisms:
$$Inn(G) := \{ i_g \, \vert \, g \in G \}.$$
So clearly the image for our homomorphism is $Inn(G)$ by definition. Next, we show that the kernel of this homomorphism is $Z(G)$, the center of $G$.
\begin{align*}
	\{ a \in G \, \vert \, i_a(x) = x \quad \forall x \in G \} &= \{  a \in G \, \vert \, axa^{-1} = x \quad \forall x \in G \} \\
	&=  \{  a \in G \, \vert \, ax = xa \quad \forall x \in G \} \\
	&= Z(G)
\end{align*}

By the First Isomorphism Theorem, we conclude that
$$G/Z(G) \cong Inn(G).$$ \qed \\

\end{document}