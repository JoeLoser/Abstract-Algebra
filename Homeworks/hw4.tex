\documentclass{article}
\usepackage{times, amsfonts}
\usepackage{amsmath, amsthm, amssymb}
\usepackage{parskip} % so I don't have to use \noindent everywhere
\usepackage{mathtools}
\usepackage[none]{hyphenat} % don't split up words that flow onto next line

\title{Abstract Algebra Homework 4}
\author{\textit{Joe Loser}}
\date{February 21, 2016}

\begin{document}
\maketitle

This problem set includes problems from sections $4.4$ and $5.3$ as well as an extra problem from lecture: in particular, problem $9$ from $4.4$, $27$ and $30$ from section $5.3$. \\

9) List every generator of each subgroup of order $8$ in $\mathbb{Z}_{32}.$

\underline{Proof}: Because $\mathbb{Z}_{32}$ is a finite cyclic group of order $32$ generated by $1$, by theorem G9, there is a \textit{unique} subgroup, let's call it $H$, of order $8$ which is 
$$ \Bigg \langle \frac{32}{8} \Bigg \rangle = \Bigg \langle 4 \cdot 1 \Bigg \rangle = \Bigg \langle 4 \Bigg \rangle. $$

To find the generators of this subgroup, we want to find all $1 \le k < 8$ such that $gcd(8, k) = 1.$ As you can see, $k$ can be $1, 3, 5,$ or $7$ (note: this is $U(8)$.) So,
$$ \langle 4^{1} \rangle = \langle 4^{3} \rangle = \langle 4^{5} \rangle = \langle 4^{7} \rangle .$$

Thus, the generators of $H$ are 
$$ \{ 4 \cdot 1, 4 \cdot 3, 4 \cdot 5, 4 \cdot 7 \} = \{4, 12, 20, 28\} $$

We can verify the order of each of these elements have order $8$:

$$ \lvert 4 \rvert = \frac{32}{gcd(4, 32)} = \frac{32}{4} = 8 $$
$$ \lvert 12 \rvert = \frac{32}{gcd(12, 32)} = \frac{32}{4} = 8 $$
$$ \lvert 20 \rvert = \frac{32}{gcd(20, 32)} = \frac{32}{4} = 8 $$
$$ \lvert 28 \rvert = \frac{32}{gcd(28, 32)} = \frac{32}{4} = 8 $$

As we have shown, the generators of $H$ (keeping in mind our binary operator is +) are $\{4, 12, 20, 28\}.$ \qed \\ \newpage

27) Let $G$ be a group and define a map $\lambda _g : G \to G$ by $\lambda _g(a) = ga.$ Prove that $\lambda _g$ is a permutation of $G.$ 

\underline{Proof}: In order to show $\lambda _g$ is a permutation of $G$, we need to show $\lambda _g(a)$ is well-defined under multiplication and that $\lambda _g$ is one-to-one and onto.

First, $\lambda _g(a)$ is well-defined since the multiplication in the group $G$ is well-defined by definition of being a group equipped with operator *.

Next, let $a, b \in G$ and $\lambda _g(a) = \lambda_g(b).$ Then we have that $ga = gb$. So, by left-cancellation, we arrive immediately at the fact that $a = b$. Hence, $\lambda _g$ is one-to-one.

Lastly, to prove $\lambda_g$ is onto, let $c \in G$. Then $g^{-1}c \in G$ since $G$ is a group ($G$ has inverses).  So, $\lambda_g(g^{-1}c) = g(g^{-1}c) = (gg^{-1})c = c.$ Hence, $\lambda_g$ is onto.

Therefore, $\lambda _g : G \to G$ is a permutation since it is one-to-one and onto. \qed \\

30) Let $\tau = (a_1, a_2, \ldots, a_k)$ be a cycle of length $k$.  \\
\begin{enumerate}
	\item[a)] Prove that if $\sigma$ is any permutation, then 
	\begin{equation}
		\sigma \tau \sigma^{-1} = (\sigma(a_1), \sigma(a_2), \ldots, \sigma(a_k))
	\end{equation}
is a cycle of length $k.$
	\item[b)] Let $\mu$ be a cycle of length $k.$ Prove that there is a permutation $\sigma$ such that 
$$ \sigma \tau \sigma^{-1} = \mu. $$
\end{enumerate}

\underline{Proof}: 
\begin{enumerate}
	\item[a)] By right multiplying equation (1) by $\sigma$, we see that equation (1)  is true precisely if
	\begin{equation}
		\sigma \tau = (\sigma(a_1), \sigma(a_2), \ldots, \sigma(a_k)) \sigma.
	\end{equation}
	
	Now, we want to prove equation (2) is indeed true. Let's consider \\ $x \not \in \{a_1, a_2, \ldots, a_k \}$ and $x\in \{a_1,a_2,\ldots,a_k\}$ for some $x$.
	
	If $x \not \in \{a_1, a_2, \ldots, a_k \}$, then $\sigma \tau (x) = \sigma(x)$ since $\tau(x) = x$. 	Further, notice that
\begin{equation}
	(\sigma(a_1), \sigma(a_2), \ldots, \sigma(a_k)) \sigma (x) = \sigma(x)
\end{equation}
because the cycle $(\sigma(a_1), \sigma(a_2), \ldots, \sigma(a_k))$ only acts on the elements $\sigma(a_i)$ for some $1 \le i \le k$. As we discussed in recitation, the cycle in (3) fixes everything else, i.e. it does not change to what it maps to. Thus, since we have $x\not\in\{a_1,\ldots,a_k\}$, the cycle fixes $\sigma(x)$. Therefore, when we consider $x \not \in \{a_1, a_2, \ldots, a_k \}$, we get the following:
\begin{align*}
	\sigma\tau (x) &= \sigma(x) \\
	&= (\sigma(a_1), \sigma(a_2), \ldots, \sigma(a_k))\sigma (x).
\end{align*}	

Now, let's consider if $x \in \{a_1, a_2, \ldots, a_k\}$. In this case, we have that $x = a_i$ for $1 \le i \le k$ as earlier. If $i \neq k$, then $\tau(a_i) = a_{i+1}$. Then, we get that
\begin{align*}
	\sigma\tau (x) &= \sigma\tau (a_i) \\
	&= \sigma(a_{i+1})
\end{align*}

So, we effectively get the next element as seen below: $$(\sigma(a_1), \sigma(a_2), \ldots, \sigma(a_k))\sigma(a_i) = \sigma(a_{i+1}).$$

If $i = k$, then $\tau(a_k) = a_1$. Hence, $\sigma\tau (a_k) = \sigma(a_1)$, i.e. it forms a k-cycle since it points back to $a_1$ and 
$$(\sigma(a_1), \sigma(a_2), \ldots, \sigma(a_k))\sigma(a_k) = \sigma(a_{1}).$$
            
Therefore $\sigma\tau (x) = (\sigma(a_1), \sigma(a_2), \ldots, \sigma(a_k)) \sigma (x)$ for all $x$ and hence equations (1) and (2) are equivalently true. \qed 
	\item[b)] Let $\mu := (b_1, b_2, \ldots, b_k)$. Now we just need to have some permutation, $\sigma$, in which  $\sigma(a_i) = b_i$ and fixes $x$ otherwise, i.e. $\sigma(x) = x$. Then by part a) of this question, we have the following:
\begin{align*}
	\sigma\tau\sigma^{-1} &= (\sigma(a_1), \sigma(a_2), \ldots, \sigma(a_k)) \\
	 &=(b_1, b_2, \ldots, b_k) \\
	 &= \mu
\end{align*}
as desired. \qed \\
\end{enumerate}

E1) Use Theorem $G9$ to dissect the group $\mathbb{Z}_{45}.$ We want to find all subgroups, list all elements of each subgroup, and find all of the generators of each subgroup. In addition, draw a subgroup diagram and show that our work implies that 
$$\sum\limits_{\substack{k \geq 1 \\ k \rvert 45}} \phi(k) = 45.$$

\underline{Solution}: \textit{All} of the generators of $\mathbb{Z}_{45}$ are the numbers $a$ that are relatively prime to $45.$ So, 
$$1, 2, 4, 7, 8, 11, 13, 14, 16, 17, 19, 22, 23, 26, 28, 29, 31, 32, 34, 37, 38, 41, 43, 44$$
are all of the generators. Note in particular that there are $24$ generators = $\phi(45)$.

To find all of the subgroups of $\mathbb{Z}_{45}$, we can just look at the positive divisors of $45$ per Theorem G9. These are $1, 3, 5, 9, 15, 45.$ Below we list the elements of each of these subgroups formed. Note that the operator in our group is addition, so the subgroups generated by $a, \langle a \rangle$, are $\{na \ \vert \ n \in \mathbb{Z} \}.$  \pagebreak

\begin{align*}
	\langle 1 \rangle &= \{1, a, 2a, 3a, \ldots, 44a \} \\
	\langle 3 \rangle &= \{1, 3a, 6a, 9a, \ldots, 42a \} \\
	\langle 5 \rangle &= \{1, 5a, 10a, 15a, 20a, 25a, 30a, 35a, 40a \} \\
	\langle 9 \rangle &= \{1, 9a, 18a, 27a, 36a \} \\
	\langle 15 \rangle &= \{1, 15a, 30a \} \\
	\langle 45 \rangle &= \{1 \} 
\end{align*}

The order of each of these subgroups is:
\begin{align*}
	\lvert \langle 1 \rangle \vert &= \frac{45}{1} = 45 \\
	\lvert \langle 3 \rangle \vert &= \frac{45}{3} = 15 \\
	\lvert \langle 5 \rangle \vert &= \frac{45}{5} = 9 \\
	\lvert \langle 9 \rangle \vert &= \frac{45}{9} = 5 \\
	\lvert \langle 15 \rangle \vert &= \frac{45}{15} = 3 \\
	\lvert \langle 45 \rangle \vert &= \frac{45}{45} = 1 \\
\end{align*}

Note: I did not feel up for learning the Tikz package for drawing the subgroup diagrams, so it is hand-drawn and attached.

To show our implies that 
$$\sum\limits_{\substack{k \geq 1 \\ k \rvert 45}} \phi(k) = 45$$
we can use brute force or think in terms of generators. The former can be seen by simply noting
\begin{align*}
	\sum\limits_{\substack{k \geq 1 \\ k \rvert 45}} \phi(k) &= \phi(1) + \phi(3) + \phi(5) + \phi(9) + \phi(15) + \phi(45) \\
	&= 1 + 2 + 4 + 6 + 8 + 24 \\
	&= 45
\end{align*}

Thinking along the line of generators, notice that if $G$ is finite cyclic group with order $n$ then 
$G$ is isomorphic to $\mathbb{Z}/n\mathbb{Z}$. If we have a generator $a \in G$ -- for example, the image of $1$ under an isomorphism of $\mathbb{Z}/n\mathbb{Z} \to G$, then $a^{k} \in G$ if and only if $gcd(n, k) = 1.$ Hence, there are exactly $\phi(n)$ generators in a finite cyclic group which is more or less the definition of the Euler-phi function. In our case, $n = 45 =  \sum\limits_{\substack{k \geq 1 \\ k \rvert 45}} \phi(k).$ \qed

\end{document}