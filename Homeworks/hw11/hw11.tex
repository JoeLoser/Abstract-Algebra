\documentclass{article}
\usepackage{amsmath, amsthm, amssymb}
\usepackage{parskip} % so I don't have to use \noindent everywhere
\usepackage{mathtools}
\usepackage[none]{hyphenat} % don't split up words that flow onto next line
\usepackage{microtype,todonotes}
\usepackage[a4paper,text={16.5cm,25.2cm},centering]{geometry}
\usepackage[compact,small]{titlesec}
\usepackage{polynom} % Polynomial long division
\setlength{\parskip}{1.2ex}
\setlength{\parindent}{0em}

% To prevent orphans and widows (https://en.wikipedia.org/wiki/Widows_and_orphans)
\clubpenalty = 10000
\widowpenalty = 10000
\usepackage{kpfonts}
\usepackage[T1]{fontenc}

\title{Abstract Algebra Homework 11}
\author{\textit{Joe Loser}}
\date{April 30, 2016}
\begin{document}
\maketitle

This problem set includes problems $3c, 4b, 24$, and an extra problem from section $17.4$.

3) Use the division algorithm to find $q(x)$ and $r(x)$ such that $a(x) = q(x)b(x) + r(x)$ with $\deg r(x) < \deg b(x)$.

3c) $a(x) = 4x^5-x^3+x^2+4$ and $b(x) = x^3-2$ where $a(x), b(x) \in \mathbb{Z}_5[x]$.

\underline{Solution}: Performing long division, we have

$$\polylongdiv{4x^5-x^3+x^2+4}{x^3-2}$$

Thus,
\begin{align*}
	a(x) &= (4x^2-1) \cdot (x^3-2) + (9x^2+2) \\
	&\equiv (4x^2-1) \cdot (x^3-2) + (4x^2+2).
\end{align*} 
\qed \\

4) Find the greatest common divisor of each of the following pairs $p(x)$ and $q(x)$ of polynomials. If \linebreak $d(x) = \gcd(p(x), q(x))$, find two polynomials $a(x)$ and $b(x)$ such that $d(x) = a(x)p(x) + b(x)q(x)$.

4b) $p(x) = x^3+x^2-x+1$ and $q(x) = x^3+x-1$ where $p(x), q(x) \in \mathbb{Z}_2[x]$.

Performing the first stage of long division, we have that
$$\polylongdiv{x^3+x-1}{x^3+x^2-x+1}$$

Note that in $\mathbb{Z}_2[x], -x^2+2x-2 \equiv -x^2$. So 
\begin{equation}
	x^3 + x -1 = 1 \cdot (x^3+x^2-x+1) + (-x^2).
\end{equation}

In the second stage of long division, we get
$$\polylongdiv{x^3+x^2-x+1}{-x^2}$$

So
\begin{equation}
	x^3+x^2-x+1 = (-x-1) \cdot (-x^2) + (-x+1).
\end{equation}

In the third stage, we get
$$\polylongdiv{-x^2}{-x+1}$$

So
\begin{equation}
	-x^2 = (-x+1) \cdot (x+1) + (-1).
\end{equation}

In the last stage of long division, we have
$$\polylongdiv{-x+1}{-1}$$

So 
\begin{equation}
	-x+1 = (-1) \cdot (x-1) + 0.
\end{equation}

Thus $\gcd(p(x), q(x)) = -1$. Performing the back substitution, we have the following \todo{finish back sub}
\begin{align*}
	-1 &= -x^2 - (x+1)(-x+1) \\
	&= -x^2 - (x+1) \big( (x^3+x^2-x+1) - (-x-1)(-x^2) \big) \\
	%&= -x^2 - (x+1) \big( (x^3+x-1-x^2) - (x+1)(-x^2) \big) \\
	%&= -x^2 - (x+1)(x^3+x-1-x^2) + 2(x+1)(-x^2) \\
	&= (x^3+x-1) - (x^3+x^2-x+1) - (x+1) \big[ (x^3+x^2-x+1) - (-x-1) \big( (x^3+x-1) - (x^3+x^2-x+1) \big) \big] \\
	%&= x^2 - (x+1) \big[ (x^3+x^2-x+1) - (-x-1)(-x^2) \big] \\
\end{align*}

24) Show that $x^p - x$ has $p$ distinct zeros in $\mathbb{Z}_p$ for any prime $p$. Conclude that 
$$x^p - x = x(x-1)(x-2) \cdots (x-(p-1)).$$

\underline{Proof}: By Fermat's Little Theorem, for all $a \in \mathbb{Z}_p$ we have that $a^p = a$. So $a^p - a = 0$. Thus every $a \in \mathbb{Z}_p$ is a zero of the polynomial $x^p - x$. Note that the polynomial has degree $p$ and $p$ zeros in $\mathbb{Z}_p$. The numbers $0, 1, \cdots, p-1$ are the roots of the equation $x^p - x$, i.e. the $p$ distinct roots. Hence it must split into $p$ distinct linear factors in $\mathbb{Z}_p[x]$ as follows:
$$x^p - x = x(x-1)(x-2) \cdots (x-(p-1)).$$ \qed \\

E1) Construct a field with $8$ elements.

\underline{Solution}: Since $8 = 2^3$ we start with a field $\mathbb{Z}_2$ of characteristic $2$ and look for an irreducible polynomial of degree $3$ in $\mathbb{Z}_2[x]$. Such a polynomial is $p(x) = x^3+x+1$. 

We will show that
$$K := \frac{\mathbb{Z}_2[x]}{<x^3+x+1>}$$
is a field of $8$ elements.

To see why $p(x)$ is irreducible in $\mathbb{Z}_2[x]$, since it of degree $3$ or lower, we can look at all of the roots in $mathbb{Z}_2$. We have that $g(0) = 1$ and $g(1) = 3 \equiv 1$. So neither $0$ or $1$ are a root of $p(x)$. Hence we see that we $p(x)$ is irreducible over $\mathbb{Z}_2[x]$.

By 

\end{document}