\documentclass{article}
\usepackage{amsmath, amsthm, amssymb}
\usepackage{parskip} % so I don't have to use \noindent everywhere
\usepackage{mathtools}
\usepackage[none]{hyphenat} % don't split up words that flow onto next line
\usepackage{microtype,todonotes}
\usepackage[a4paper,text={16.5cm,25.2cm},centering]{geometry}
\usepackage[compact,small]{titlesec}
\usepackage{polynom} % Polynomial long division
\setlength{\parskip}{1.2ex}
\setlength{\parindent}{0em}

% To prevent orphans and widows (https://en.wikipedia.org/wiki/Widows_and_orphans)
\clubpenalty = 10000
\widowpenalty = 10000
\usepackage{kpfonts}
\usepackage[T1]{fontenc}

\title{Abstract Algebra Homework 11}
\author{\textit{Joe Loser}}
\date{April 30, 2016}
\begin{document}
\maketitle

This problem set includes problems $3c, 4b, 24$, and an extra problem from section $17.4$. \\

3) Use the division algorithm to find $q(x)$ and $r(x)$ such that $a(x) = q(x)b(x) + r(x)$ with $\deg r(x) < \deg b(x)$.

3c) $a(x) = 4x^5-x^3+x^2+4$ and $b(x) = x^3-2$ where $a(x), b(x) \in \mathbb{Z}_5[x]$.

\underline{Solution}: First note that in $\mathbb{Z}_5[x], a(x) \equiv 4x^5+4x^3+x^2+4$ and $b(x) \equiv x^3 + 3$.

Now performing long division, we have

$$\polylongdiv{4x^5+4x^3+x^2+4}{x^3+3}$$

Thus,
\begin{align*}
	a(x) &= (4x^2+4) \cdot (x^3+3) + (-11x^2-8) \\
	&\equiv (4x^2+4) \cdot (x^3+3) + (4x^2+2).
\end{align*} 
\qed \\

4) Find the greatest common divisor of each of the following pairs $p(x)$ and $q(x)$ of polynomials. If \linebreak $d(x) = \gcd(p(x), q(x))$, find two polynomials $a(x)$ and $b(x)$ such that $d(x) = a(x)p(x) + b(x)q(x)$.

4b) $p(x) = x^3+x^2-x+1$ and $q(x) = x^3+x-1$ where $p(x), q(x) \in \mathbb{Z}_2[x]$.

First note that $\mathbb{Z}_2[x]$, $p(x) \equiv x^3 + x^2 + x + 1$ and $q(x) \equiv x^3 + x + 1$.

Performing the first stage of long division, we have that
$$\polylongdiv{x^3+x+1}{x^3+x^2+x+1}$$

Note that in $\mathbb{Z}_2[x], -x^2 \equiv x^2$ and $2x^2 \equiv 0$. So 
\begin{equation}
	x^3 + x + 1 = 1 \cdot (x^3+x^2+x+1) + (x^2).
\end{equation}

In the second stage of long division, we get
$$\polylongdiv{x^3+x^2+x+1}{x^2}$$

So we have that
\begin{equation}
	x^3+x^2+x+1 = (x+1) \cdot (x^2) + (x+1).
\end{equation}

In the third stage, we get
$$\polylongdiv{x^2}{x+1}$$

Then
\begin{equation}
	x^2 = (x+1) \cdot (x+1) + (1).
\end{equation}

In the last stage of long division, we have
$$\polylongdiv{x+1}{1}$$

So 
\begin{equation}
	x+1 = (1) \cdot (x+1) + 0.
\end{equation}

Thus $\gcd(p(x), q(x)) = 1$. Performing the back substitution, we have the following 
\begin{align*}
	1 &= x^2 - (x+1)(x+1) \\
	&= x^2 - (x+1) [(x^3+x^2+x+1) - x^2(x+1)] \\
	&= x^2 + (x+1)(x^3+x^2+x+1) + x^2(x+1)^2 \\
	&= x^2 \big(1+(x+1)^2 \big) + (x+1)(x^3+x^2+x+1) \\
	&= [x^3+x+1 + (x^3+x^2+x+1)](x^2) + (x+1)(x^3+x^2+x+1) \\
	&= (x^3+x^2+x+1)(x^2+x+1) + (x^2)(x^3+x+1).
\end{align*}

Thus
\begin{align*}
	1 &= (x^2+x+1)(x^3+x^2+x+1) + (x^2)(x^3+x+1) \\
	&= a(x)p(x) + b(x)q(x).
\end{align*}

To check this holds in $\mathbb{Z}_2[x]$, we multiply everything out. We get that
\begin{align*}
	1 &= (x^2+x+1)(x^3+x^2+x+1) + (x^2)(x^3+x+1) \\
	&= x^5 + x^4 + x^3 + x^2 + x^4 + x^3 + x^2 + x + x^3 + x^2 + x + 1 + x^5 + x^3 + x^2 \\
	&= 2x^5 + 2x^4 + 4x^3 + 4x^2 + 2x + 1 \\
	&\equiv 1.
\end{align*} 
\qed \\

\pagebreak 
24) Show that $x^p - x$ has $p$ distinct zeros in $\mathbb{Z}_p$ for any prime $p$. Conclude that 
$$x^p - x = x(x-1)(x-2) \cdots (x-(p-1)).$$

\underline{Proof}: By Fermat's Little Theorem, for all $a \in \mathbb{Z}_p$ we have that $a^p = a$. So $a^p - a = 0$. Thus every $a \in \mathbb{Z}_p$ is a zero of the polynomial $x^p - x$. Note that the polynomial has degree $p$ and $p$ zeros in $\mathbb{Z}_p$. The numbers $0, 1, \cdots, p-1$ are the roots of the equation $x^p - x$, i.e. the $p$ distinct roots. Hence it must split into $p$ distinct linear factors in $\mathbb{Z}_p[x]$ as follows:
$$x^p - x = x(x-1)(x-2) \cdots (x-(p-1)).$$ \qed \\

E1) Construct a field with $8$ elements.

\underline{Solution}: Since $8 = 2^3$ we start with a field $\mathbb{Z}_2$ of characteristic $2$ and look for an irreducible polynomial of degree $3$ in $\mathbb{Z}_2[x]$. Such a polynomial is $p(x) = x^3+x+1$. 

We will show that
$$K := \frac{\mathbb{Z}_2[x]}{\langle x^3+x+1 \rangle}$$
is a field of $8$ elements.

To see why $p(x)$ is irreducible in $\mathbb{Z}_2[x]$, since it of degree $3$ or lower, we can look at all of the roots in $\mathbb{Z}_2$ since $g(x)$ is irreducible if and only if $p$ does not have a root, i.e. $p(a) \neq 0$ for all $a \in \mathbb{Z}_2$. We have that $g(0) = 1$ and $g(1) = 3 \equiv 1$. So neither $0$ or $1$ are a root of $p(x)$. Hence we see that we $p(x)$ is irreducible over $\mathbb{Z}_2[x]$.

By the Division Algorithm, we have that
$$p(x) + \langle x^3+x+1 \rangle =  a_0 + a_1 x + a_2 x^2 + \langle x^3+x+1 \rangle $$
where $a_0, a_1, a_2 \in \mathbb{Z}_2$. So 
$$ K = \{ a_0 + a_1 x + a_2 x^2 + \langle x^3+x+1 \rangle \, \vert \, a_0, a_1, a_2 \in \mathbb{Z}_2 \}. $$

Note that since $a_0, a_1, a_2 \in \mathbb{Z}_2$ we see that the order of $K$ which we denote $\lvert K \rvert$ is
\begin{align*}
	\lvert K \rvert &\leq \lvert \mathbb{Z}_2 \rvert \times \lvert \mathbb{Z}_2 \rvert \times \lvert \mathbb{Z}_2 \rvert \\
	&= 2 \times 2 \times 2 \\
	&= 8.
\end{align*}

To see why $K$ has exactly $8$ elements, suppose $a_0 + a_1x + a_2x^2 + \langle x^3 + x + 1 \rangle = b_0 + b_1 x + b_2 x^2 + \langle x^3+x+1 \rangle$. Then 
$$(a_0 - b_0) + (a_1-b_1)x + (a_2-b_2)x^2 \in \lvert x^3+x+1 \rvert. $$
That is,
$$ \underbrace{x^3 + x + 1}_{\deg 3} \, \vert \, \underbrace{(a_0-b_0) + (a_1-b_1)x + (a_2-b_2)x^2}_{\deg < 3}. $$
So $(a_0-b_0) + (a_1-b_1)x + (a_2-b_2)x^2 = 0$. Hence $a_0 = b_0, a_1 = b_1$ and $a_2 = b_2$. Therefore $\lvert K \rvert = 8$.

To explicitly see what the elements of $K$ are, let $\alpha := x + \langle x^3+x+1 \rangle \in K$. Now we want to compute the powers of $\alpha$ for $i = 1 \cdots 7$. This will give us the elements of $K$.
\begin{align*}
	\alpha^2 &= (x + \langle x^3 + x + 1\rangle)^2 \\
	&= x^2 + 2x \langle x^3 + x + 1\rangle + (\langle x^3 + x + 1\rangle)^2 \\
	&\equiv x^2 + \langle x^3 + x + 1\rangle.
\end{align*}

In $K$, $x^3 + x + 1 = 0 \iff \alpha^3 + \alpha + 1 = 0 \iff \alpha^3 = \alpha + 1$. That is, $\alpha^3 = x + 1 + \langle x^3 + x + 1\rangle$. Next we see that
\begin{align*}
	\alpha^4 &= \alpha^3 \cdot \alpha \\
	&= (\alpha + 1) \cdot \alpha \\
	&= (x+1) \cdot x \\
	&\equiv x^2 + x + \langle x^3 + x + 1\rangle.
\end{align*}

Also 
\begin{align*}
	\alpha^5 &= \alpha^3 \cdot \alpha^2 \\
	&= (\alpha + 1) \cdot \alpha^2 \\
	&= \alpha^3 + \alpha^2 \\
	&= (\alpha + 1) + \alpha^2 \\
	&\equiv x + 1 + x^2 \\
	&\equiv x^2 + x + 1 + \langle x^3 + x + 1\rangle.
\end{align*}

Continuing on, we have that
\begin{align*}
	\alpha^6 &= \alpha^3 \cdot \alpha^3 \\
	&= (x+1)(x+1) \\
	&= x^2 + 2x + 1 \\
	&\equiv x^2 + 1 + \langle x^3 + x + 1\rangle.
\end{align*}

Lastly we have that 
\begin{align*}
	\alpha^7 &= \alpha^4 \cdot \alpha^3 \\
	&= (\alpha^2 + \alpha) \cdot (\alpha + 1) \\
	&= \alpha^3 + \alpha^2 + \alpha^2 + \alpha \\
	&= \alpha^3 + \alpha \\
	&= (\alpha + 1) + \alpha \\
	&\equiv 1.
\end{align*}

Explicitly listing the elements of $K$, we have that
\begin{align*}
	K &= \{0, x, x^2, x + 1, x^2 + x, x^2 + x + 1, x^2 + 1, 1 \} \\
	&= \{ 0, 1, x, x+1, x^2, x^2+1, x^2+x, x^2+x+1 \}.
\end{align*}
\qed \\

\end{document}