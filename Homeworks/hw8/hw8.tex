\documentclass{article}
\usepackage{amsmath, amsthm, amssymb}
\usepackage{parskip} % so I don't have to use \noindent everywhere
\usepackage{mathtools}
\usepackage[none]{hyphenat} % don't split up words that flow onto next line
\usepackage{microtype,todonotes}
\usepackage[a4paper,text={16.5cm,25.2cm},centering]{geometry}
\usepackage[compact,small]{titlesec}
\setlength{\parskip}{1.2ex}
\setlength{\parindent}{0em}
\DeclareMathOperator{\lcm}{lcm}

% To prevent orphans and widows (https://en.wikipedia.org/wiki/Widows_and_orphans)
\clubpenalty = 10000
\widowpenalty = 10000
\usepackage{kpfonts}
\usepackage[T1]{fontenc}

\title{Abstract Algebra Homework 8}
\author{\textit{Joe Loser}}
\date{April 3, 2016}
\begin{document}
\maketitle

This problem set includes problems $2, 24, 28, 34,$ and $38$ from section $16.6$.

2) Let $R$ be the ring of $2 \times 2$ matrices of the form
$$ \begin{pmatrix}
	a & b \\
	0 & 0
\end{pmatrix},$$ 
where $a, b \in \mathbb{R}$. Show that although $R$ is a ring that has no identity, we can find a subring $S$ of $R$ with an identity.

\underline{Proof}: We first show that $R$ has no identity. Suppose for the sake of contradiction that $R$ has an identity which we denote $1_R$. From the definition of $R$ we see that $1_R \in R$ means there exist $a, b \in \mathbb{R}$ such that $1_R = \begin{pmatrix}
	a & b \\
	0 & 0
\end{pmatrix}$. Since the element
$\begin{pmatrix}
	1 & 0 \\
	0 & 0
\end{pmatrix} \in R$ and $1_R$ is the identity for $R$ we have that
$$
	\begin{pmatrix}
	1 & 0 \\
	0 & 0
	\end{pmatrix} \cdot 1_R = 
	\begin{pmatrix}
	1 & 0 \\
	0 & 0
	\end{pmatrix}.
$$
But from the definition of multiplication in $R$ we also have that 
$$
	\begin{pmatrix}
	1 & 0 \\
	0 & 0
	\end{pmatrix} \cdot 1_R = 
	\begin{pmatrix}
	a & b \\
	0 & 0
	\end{pmatrix}.
$$
Thus we see that $a = 1, b = 0$. This means that $$1_R = 
\begin{pmatrix}
1 & 0 \\
0 & 0
\end{pmatrix}.$$ However, this element $1_R$ is clearly not the identity for $R$. For instance, consider another element, say $
\begin{pmatrix}
	0 & 1 \\
	0 & 0
\end{pmatrix} \in R$. We have that 
\begin{equation}
	1_R \cdot 
	\begin{pmatrix}
		0 & 1 \\
		0 & 0
	\end{pmatrix} =
	\begin{pmatrix}
		0 & 1\\
		0 & 0
	\end{pmatrix}.
\end{equation}
Also,
\begin{equation}
	\begin{pmatrix}
	0 & 1 \\
	0 & 0
	\end{pmatrix}
	\cdot 1_R =
	\begin{pmatrix}
	0 & 0\\
	0 & 0
	\end{pmatrix}.
\end{equation}

Since the right hand side of both $(1)$ and $(2)$ do not agree, we see that $1_R$ cannot be the identity for $R$ which is a contradiction. Thus $R$ is a ring without an identity element.

Even though $R$ has no identity element, we can find a subring $S$ of $R$ which has an identity. We claim that $S =
\begin{pmatrix}
	a & 0 \\
	0 & 0
\end{pmatrix}$ where $a \in \mathbb{R}$ is a subring of $R$. We will use the Subring Test to show that this is indeed a subring.

i) We first see that $S$ is clearly nonempty since $a$ is any real number.

ii) We now show that $rs \in S$ for all $r, s \in S$. Let $r = 
\begin{pmatrix}
	a & 0 \\
	0 & 0
\end{pmatrix}$ and $s =
\begin{pmatrix}
	b & 0 \\
	0 & 0
\end{pmatrix}$ where $a, b \in \mathbb{R}$. Then we have that 
\begin{align*}
	rs &= 
	\begin{pmatrix}
		a & 0 \\
		0 & 0 \\
	\end{pmatrix}
	\begin{pmatrix}
		b & 0 \\
		0 & 0 \\
	\end{pmatrix} \\
	&= \begin{pmatrix}
		ab & 0 \\
		0 & 0
	\end{pmatrix} \\
	& \in S
\end{align*} since $ab \in \mathbb{R}$ because $a$ and $b$ are both in $\mathbb{R}$. 

iii) Lastly, we show that $r - s \in S$ for all $r, s \in S$. Let $r$ and $s$ be as above in ii). Then
\begin{align*}
	r - s &=
	\begin{pmatrix}
		a - b & 0 \\ 
		0 & 0 \\
	\end{pmatrix} \\
	& \in S
\end{align*} since $a - b \in \mathbb{R}$ because both $a$ and $b$ are in $\mathbb{R}$.

Thus we have shown that $S =
\begin{pmatrix}
a & 0 \\
0 & 0
\end{pmatrix}$ is indeed a subring of $R$ by the Subring Test. \qed \\

24) Let $R$ be a ring with a collection of subrings $\{R_{\alpha}\}$. Prove that $\bigcap R_{\alpha}$ is a subring of $R$. Give an example to show that the union of two subrings need not be a subring.

\underline{Proof}: Let $S$ be the intersection of a collection of subrings of the ring $R$. That is, $S = \underset{i \in I}{\bigcap} S_i$ where $I$ is an indexed set and each $S_i$ is a subring of $R$. We will use the Subring Test to show that $S$ is indeed a subring of $R$.

We first begin with a claim and its proof to use it later on.

\underline{Claim}: If $S$ is a subring of a ring $R$ then $0 \in S$.

\underline{Proof of Claim}: If $S$ is a subring of $R$ then $S$ is nonempty. Let $x \in S$. Then since $S$ is a ring and has closure under additive inverses and addition, we have that $x + (-x) \in S$. By definition of additive inverses, $x + (-x) = 0$. Thus $0 \in S$. \pagebreak

We now check the conditions of the Subring Test hold.

i) To show that $S$ is nonempty, just apply the result from the claim. Since $0 \in S_i$ for each $i \in I$ we have that $0 \in \underset{i \in I}{\bigcap} S_i$; i.e. $0 \in S$.

ii) Next we show for all $a, b \in S, a - b \in S$. Let $a, b \in S$. By definition of $S$ we see $a, b \in S_i$ for each $i \in I$. By assumption that each $S_i$ is a subring (and so $S_i$ is a ring), we have that $a - b \in S_i$ for each $i \in I$. Then by definition of intersection, this means that $a - b \in \underset{i \in I}{\bigcap} S_i$ for each $i \in I$. That is, $a - b \in S$.

iii) Lastly we show that for all $a, b \in S, ab \in S$. Let $a, b \in S$. By definition of $S$ we see that $a, b \in S_i$ for each $i \in I$. By assumption that each $S_i$ is a subring (and so $S_i$ is a ring), we have that $ab \in S_i$ for each $i \in I$. Then by definition of intersection, this means that $ab \in \underset{i \in I}{\bigcap} S_i$ for each $i \in I$. That is, $ab \in S$.

Thus $S$ is a subring by the Subring Test.

To give an example to show that the union of two subrings need not be a subring, consider the following:
$$R = \mathbb{Z} \quad S = \{2n \, \vert \, n \in \mathbb{Z} \} \quad T = \{3n \, \vert \, n \in \mathbb{Z} \}.$$
Note that $R$ is a ring and $S$ and $T$ are subrings of $R$ (one can easily verify this -- see Example 16.9 in the text). We will show that $S \cup T$ is not a subring of $R$. Consider two elements: $2 \in S, 3 \in T$. Clearly both are in $S \cup T$. However $2 + 3 = 5 \not \in S \cup T$. So $S \cup T$ is not a ring (and hence not a subring of R). \qed

28) A ring $R$ is a Boolean ring if for every $a \in \mathbb{R}, a^2 = a$. Show that every Boolean ring is a commutative ring.

\underline{Proof}: We know that $R$ is a commutative ring if $ab = ba$ for all $a, b \in R$.

Let $a, b \in R$. Notice that since $R$ is a Boolean ring and $a, b \in R$ we have that
\begin{align*}
	a + b &= (a+b)^2 \\
	&= (a+b)(a+b) \\
	&= a(a+b) + b(a+b) \\
	&= a^2 + ab + ba + b^2 \\
	&= a + ab + ba + b^2 \quad \because a^2 = a \quad \text{and} \quad b^2 = b.
\end{align*}
By subtracting $a + b$ from both sides we have that $0 = ab + ba$. So $-ab = ba$. We are almost done since we want to show that $ab = ba$. To conclude, we will show that for all $c \in R$, $-c = c$. Let $c \in R$. Then
\begin{align*}
	-c &= (-c)^2 \\
	&= (-c)(-c) \\
	&= -c(-c) \\
	&= -(-c^2) \\
	&= c^2 \\
	&= c \quad \because \text{R is boolean}.
\end{align*}

Thus $-ab = ba \implies ab = ba$ since $-c = c$ for all $c \in R$ and both $a$ and $b$ are arbitrary elements in $R$ as well. \qed \\

34) Let $p$ be prime. Prove that
$$Z_{(p)} = \big\{ \frac{a}{b} \big\vert \, a, b \in \mathbb{Z} \, \text{and} \, \gcd(b, p) = 1 \big\} $$
is a ring. 

\underline{Proof}: To show that $Z_{(p)}$ is a ring, we can verify directly by checking all of the properties of a ring using the definition of a ring. Or, better yet, we can show that $Z_{(p)}$ is a subring of a known ring and hence is a ring itself. We will show the latter.

Notice that as sets, $Z_{(p)} \subset \mathbb{Q}$ and $\mathbb{Q}$ is a well-known ring. We will show that $Z_{(p)}$ is a subring of $\mathbb{Q}$ by using the Subring Test.

i) To show that $Z_{(p)}$ is nonempty, simply take $a = 1, b = 1$ which is an element of $Z_{(p)}$ since $\gcd(b, p) = 1$ for any $p$ prime.

ii) Next we show that for all $r, s \in Z_{(p)}, rs \in Z_{(p)}$. Let $r, s \in Z_{(p)}$ So $r = \frac ab, s = \frac cd$ for some $a, b, c, d \in \mathbb{Z}$ and $\gcd(b, p) = \gcd(d, p) = 1$. Then we have that
\begin{align*}
	rs &= \frac ab \cdot \frac cd \\
	&= \frac{ac}{bd}.
\end{align*}
Notice that $ac \in \mathbb{Z}, bd \in \mathbb{Z}$. Also $\gcd(bd, p) = 1$ since $\gcd(b, p) = \gcd(d, p) = 1$. Thus $rs = \frac{ab}{cd} \in Z_{(p)}$.

iii) Lastly we show that for all $r, s \in Z_{(p)}, r - s \in Z_{(p)}$. Let $r, s \in Z_{(p)}$ as before in ii). Then 
\begin{align*}
	r - s &= \frac ab - \frac cd \\
	&= \frac{ad-bc}{bd}.
\end{align*}
Notice that $ad - bc \in \mathbb{Z}, bd \in \mathbb{Z}$ and $\gcd(bd, p) = 1$ since $\gcd(b, p) = \gcd(d, p) = 1$. Thus $\frac{ad-bc}{bd} \in Z_{(p)}.$

By the Subring Test, we conclude that $Z_{(p)}$ is a subring of $\mathbb{Q}$ and so $Z_{(p)}$ is a ring. \qed \\

38) An element $x$ in a ring is called idempotent if $x^2 = x$. Prove that the only idempotent in an integral domain are $0$ and $1$. Find a ring with an idempotent $x$ not equal to $0$ or $1$. 

\underline{Proof}: Let $R$ be an integral domain and $x \in R$ be an idempotent element. Then 
$$ x^2 = x \implies x^2 - x = 0 \implies x(x-1) = 0.$$
Since $R$ is an integral domain, there are no zero divisors. Thus $x = 0$ or $x-1 = 0$. So the only idempotents are $0$ and $1$. 

To give an example of a ring with an idempotent $x$ not equal to $0$ or $1$, consider the ring $\mathbb{Z}_{12}$. Continually squaring elements in $\pmod {12}$ we have that $0^2 \equiv 0, 1^2 \equiv 1, 2^2 \equiv 4, 3^2 \equiv 9, 4^2 \equiv 4, 5^2 \equiv 1, 6^2 \equiv 0, 7^2 \equiv 1, \linebreak 8^2 \equiv 4, 9^2 \equiv 9, 10^2 \equiv 4, 11^2 \equiv 1$. So in $\mathbb{Z}_{12}$ the idempotent elements are $0, 1, 4,$ and $9$. So we have found a ring with idempotent elements other than the trivial ones of $0$ and $1$ so we are done. \qed \\

\end{document}