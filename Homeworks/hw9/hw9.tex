\documentclass{article}
\usepackage{amsmath, amsthm, amssymb}
\usepackage{parskip} % so I don't have to use \noindent everywhere
\usepackage{mathtools}
\usepackage[none]{hyphenat} % don't split up words that flow onto next line
\usepackage{microtype,todonotes}
\usepackage[a4paper,text={16.5cm,25.2cm},centering]{geometry}
\usepackage[compact,small]{titlesec}
\setlength{\parskip}{1.2ex}
\setlength{\parindent}{0em}

% For addition and multiplication tables
\usepackage{dcolumn}
\newcolumntype{2}{D{.}{}{2.0}}

% To prevent orphans and widows (https://en.wikipedia.org/wiki/Widows_and_orphans)
\clubpenalty = 10000
\widowpenalty = 10000
\usepackage{kpfonts}
\usepackage[T1]{fontenc}

\title{Abstract Algebra Homework 9}
\author{\textit{Joe Loser}}
\date{April 11, 2016}
\begin{document}
\maketitle

This problem set includes problems $5b, 8, 18c$, and $27$ from section $16.6$.

5b) For the given ring $R$ with an ideal $I$, give an addition and multiplication table for $R/I$.

\underline{Solution}: Recall that 
$$R/I := \{ r+I \, \vert \, r \in R \}.$$

We can easily see that the three elements of $R/I$ are the following:
\begin{align*}
	0 + I &= \{ 0, 3, 6, 9 \} \\
	1 + I &= \{ 1, 4, 7, 10 \} \\
	2 + I &= \{ 2, 5, 8, 11 \}.
\end{align*}

Below is the addition table for $R/I$. Note that it is implicit, but worth noting, that we are talking about the addition of the three cosets here in $R/I$.

\begin{center}
	\begin{tabular}{r|*{4}{2|}}
		+ & 0 & 1 & 2 \\
		\hline
		0 & 0 & 1 & 2 \\ 
		\hline
		1 & 1 & 2 & 0  \\ 
		\hline
		2 & 2 & 0 & 1
	\end{tabular}
\end{center}

Similarly, here is the multiplication table for $R/I$. Again, we are talking about multiplication of the cosets in $R/I$.

\begin{center}
	\begin{tabular}{r|*{4}{2|}}
		* & 0 & 1 & 2 \\
		\hline
		0 & 0 & 0 & 0 \\ 
		\hline
		1 & 0 & 1 & 2  \\ 
		\hline
		2 & 0 & 2 & 1
	\end{tabular}
\end{center}
\qed \\

\end{document}