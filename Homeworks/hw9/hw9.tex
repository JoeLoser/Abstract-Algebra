\documentclass{article}
\usepackage{amsmath, amsthm, amssymb}
\usepackage{parskip} % so I don't have to use \noindent everywhere
\usepackage{mathtools}
\usepackage[none]{hyphenat} % don't split up words that flow onto next line
\usepackage{microtype,todonotes}
\usepackage[a4paper,text={16.5cm,25.2cm},centering]{geometry}
\usepackage[compact,small]{titlesec}
\setlength{\parskip}{1.2ex}
\setlength{\parindent}{0em}

% For addition and multiplication tables
\usepackage{dcolumn}
\newcolumntype{2}{D{.}{}{2.0}}

% To prevent orphans and widows (https://en.wikipedia.org/wiki/Widows_and_orphans)
\clubpenalty = 10000
\widowpenalty = 10000
\usepackage{kpfonts}
\usepackage[T1]{fontenc}

\title{Abstract Algebra Homework 9}
\author{\textit{Joe Loser}}
\date{April 11, 2016}
\begin{document}
\maketitle

This problem set includes problems $5b, 8, 18c$, and $27$ from section $16.6$.

5b) For the given ring $R$ with an ideal $I$, give an addition and multiplication table for $R/I$.

\underline{Solution}: Recall that 
$$R/I := \{ r+I \, \vert \, r \in R \}.$$

We can easily see that the three elements of $R/I$ are the following:
\begin{align*}
	0 + I &= \{ 0, 3, 6, 9 \} \\
	1 + I &= \{ 1, 4, 7, 10 \} \\
	2 + I &= \{ 2, 5, 8, 11 \}.
\end{align*}

Below is the addition table for $R/I$. Note that it is implicit, but worth noting, that we are talking about the addition of the three cosets here in $R/I$.

\begin{center}
	\begin{tabular}{r|*{4}{2|}}
		+ & 0 & 1 & 2 \\
		\hline
		0 & 0 & 1 & 2 \\ 
		\hline
		1 & 1 & 2 & 0  \\ 
		\hline
		2 & 2 & 0 & 1
	\end{tabular}
\end{center}

Similarly, here is the multiplication table for $R/I$. Again, we are talking about multiplication of the cosets in $R/I$.

\begin{center}
	\begin{tabular}{r|*{4}{2|}}
		* & 0 & 1 & 2 \\
		\hline
		0 & 0 & 0 & 0 \\ 
		\hline
		1 & 0 & 1 & 2  \\ 
		\hline
		2 & 0 & 2 & 1
	\end{tabular}
\end{center}
\qed \\

8) Prove or disprove: The ring $\mathbb{Q}(\sqrt{2}) = \{ a+b\sqrt{2} \, \vert \, a, b \in \mathbb{Q} \}$ is isomorphic to the ring $\mathbb{Q}(\sqrt{3}) = \{ a+b\sqrt{3} \, \vert \, a, b \in \mathbb{Q} \}$.

\underline{Proof}: We will show that $\mathbb{Q}(\sqrt{2}) \not \cong \mathbb{Q}(\sqrt{3})$. To do this, we need to show that no homomorphism from $\mathbb{Q}(\sqrt{2})$ to $\mathbb{Q}(\sqrt{3})$ can be an isomorphism.

Suppose that $\phi : \mathbb{Q}(\sqrt{2}) \to \mathbb{Q}(\sqrt{3})$ is a homomorphism. We will begin by showing that $\phi$ fixes $\mathbb{Z}$ and $\mathbb{Q}$.

We first show that $\phi(1) = 1$ since we do not get this for free by our definition of a ring homomorphism. Let $x \in \mathbb{Z}$. Suppose $\phi(1) = a + b\sqrt{3}$ where $a, b \in \mathbb{Q}$. Then we have that
\begin{align*}
	\phi(x) &= \phi(x \cdot 1) \\
	&= \phi(x) \phi(1) \\
	&= \phi(x \dot 1) \phi(1) \\
	&= \phi(x) \phi(1)^2 \\
	&= \vdots \\
	&= \phi(x) \phi(1)^n.
\end{align*}
So $\phi(x) = \phi(x) \phi(1)^n$. Thus $1 = \phi(1)^n$. Hence $\phi(1) = 1$. 

We now use this result of $\phi(1) = 1$ to show that $\phi$ fixes $\mathbb{Z}$. That is, we extend it to show $\phi(n) = n$ for all $n \in \mathbb{Z}$. To show this, let $n \in \mathbb{Z^+}$. Then
\begin{align*}
	\phi(n) &= \phi \underbrace{(1 + \cdots + 1)}_{n \text {-times}} \\
	&= \underbrace{\phi(1) + \cdots + \phi(1)}_{n \text {-times}} \\
	&= \underbrace{1 + \cdots + \cdots 1}_{n \text {-times}} \\
	&= n.
\end{align*}
The proof is similar for $n \in \mathbb{Z^-}$ and we conclude that $\phi(n) = n$ for all $n \in \mathbb{Z}$.

Next, we show that $\phi$ fixes $\mathbb{Q}$. That is, $\phi(y) = y$ for all $y \in \mathbb{Q}$. Let $y = \frac ab$ where $a, b \in \mathbb{Z}$. Then 
\begin{align*}
	\phi(y) &= \phi\big(\frac ab\big) \\
	&= \phi(ab^{-1}) \\
	&= \phi(a) \phi(b^{-1}) \\
	&= \frac{\phi(a)}{\phi(b)} \\
	&= \frac ab
\end{align*}
since we just showed $\phi(n) = n$ for all $n \in \mathbb{Z}$. Thus $\phi(y) = y$ for all $y \in \mathbb{Q}$.

Therefore, if $a + b\sqrt{2} \in \mathbb{Q}(\sqrt{2})$ we see that 
\begin{align*}
	\phi(a + b\sqrt{2}) &= \phi(a) + \phi(b\sqrt{2}) \\
	&= \phi(a) + \phi(b) \phi(\sqrt{2}) \\
	&= a + b \phi(\sqrt{2}).
\end{align*}
So we need to figure out what exactly $\phi(\sqrt{2})$ is.

If $\phi(\sqrt{2}) = c + d\sqrt{3}$ for some $c, d \in \mathbb{Q}$ then what are the possible values of $c$ and $d$? First notice that $\phi(2) = 2$ since $2 \in \mathbb{Z}$. We also know that $2 = (\sqrt{2})^2$. So then 
\begin{align*} 
	\phi(2)  &= \phi \big( (\sqrt{2})^2 \big) \\
	&= \phi(\sqrt{2}) \phi(\sqrt{2}) \\
	&= (c+d\sqrt{3})^2 \\
	&= c^2 + 3d^2 + 2cd\sqrt{3}.
\end{align*}
Thus $2 = c^2 + 3d^2$ and $0 = 2cd\sqrt{3}$. Hence $cd = 0$. So either $c = 0$ or $d = 0$. 

If $c = 0$ then we have $2 = 3d^2$. So $d = \sqrt{\frac 23} \not \in \mathbb{Q}$ which is a contradiction since $d \in \mathbb{Q}$. Similarly, if $d = 0$ then we have that $2 = c^2$. So $c = \sqrt 2 \not \in \mathbb{Q}$ which is a contradiction since $c \in \mathbb{Q}$.

Thus $\phi(\sqrt{2}) = c + d\sqrt{3}$ for $c, d \in \mathbb{Q}$. But we have shown that we cannot find a suitable $c$ or $d \in \mathbb{Q}$ such that this is satisfied. Hence we have shown that there is no isomorphism from $\mathbb{Q}(\sqrt{2})$ to $\mathbb{Q}(\sqrt{3})$. Therefore, as rings, $\mathbb{Q}(\sqrt{2}) \not \cong \mathbb{Q}(\sqrt{3})$. \qed \\

\end{document}